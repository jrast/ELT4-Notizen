\section{Reaktanz Eintore}
\index{Reaktanz Eintore|see{RET}}
\index{RET}
Ein Reaktanz Eintor ist eine Schaltung aus lauter $L$ und $C$ mit genau zwei
Anschlüssen.

Skript von Herrn Neubauer, Seiten 27ff

Bei $\omega = 0$ und $\omega \rightarrow \infty$ hat $X(\omega)$ entweder einen
Pol oder eine Nullstelle.


\subsection{RET-Typen}
\index{RET!Typen}
Man unterscheidet die Reaktanzeintore in vier Typen. Dazu betrachtet man das
Verhalten für $\omega \rightarrow 0$ und $\omega \rightarrow \infty$. Mit diesen
beiden Angaben kann der Typ mit der nachfolgenden Tabelle bestimmt werden.

\begin{table}[!h]
\begin{center}
\begin{tabular}{|l|l|c|l|}
  \hline
  $\bf \boldsymbol\omega = 0$ & $\bf \boldsymbol\omega \rightarrow
  \boldsymbol\infty$ & \textbf{Typ} & \textbf{Symbol} \\
  \hline
  Nullstelle & Polstelle  & L & L-Symbol \\
  \hline
  Polstelle  & Nullstelle & C & C-Symbol \\
  \hline
  Polstelle  & Polstelle  & S & LC-Serie \\
  \hline
  Nullstelle & Nullstelle & P & LC-Parallel \\
  \hline
\end{tabular}
\caption{Bestimmung des RET-Typ}
\label{tab:RETTypBestimmung}
\end{center}
\end{table}

$$\underline{Z}(p)=\frac{a_np^n+a_{n-2}p^{n-2}+\ldots}{b_mp^m+b_{m-2}p^{m-2}}$$\\
$p\rightarrow \infty$ höchsten Potenzen entscheidend\\
\hspace*{5pt}$n>m$ (Zählergrad $>$ Nennergrad): Pol\\
\hspace*{10pt}$\frac{a_n}{b_m}=L_{\infty}$\\
\hspace*{5pt}$n<m$ (Zählergrad $<$ Nennergrad): Nullstelle\\
\hspace*{10pt}$\frac{a_n}{b_m}=\frac{1}{C_{\infty}}$\\
$p\rightarrow 0$ tiefste Potenzen massgebend\\
\hspace*{5pt}n ungerade
$\frac{\ldots+a_3p^3+a_1p}{\ldots+b_2p^2+b_0}\rightarrow$ Nullstelle für $p\rightarrow 0$\\
\hspace*{10pt}$\frac{a_1}{b} = L_0$ ersetzung bei ganz tiefen Frequenzen.\\
\hspace*{5pt}n gerade $\frac{\ldots+a_2p^2+a_0}{\ldots+b_3p^3+b_1p}$\\
\hspace*{10pt}$\frac{a_0}{b_1}=\frac{1}{C_0}$ ersetzung bei ganz tiefen
Frequenzen.\\


\subsection{Bestimmung des RET-Typs anhand der Schaltung}
\begin{tabular}{llll}
\begin{circuitikz}[scale=2, european, american inductors, yscale=0.4]
\ctikzset{bipoles/length=1.2cm}
\draw (0,0)
	to[L, *-*] (2,0)
	;	
\end{circuitikz}
 & L-Typ & L-Klemmenkreis & L-Klemmentrennbündel\\
& & Nullstelle bei $\omega = 0$ & Pol bei $\omega \rightarrow \infty$ \\
\begin{circuitikz}[scale=2, european, american inductors, yscale=0.4]
\ctikzset{bipoles/length=1.2cm}
\draw (0,0)
	to[C, *-*] (2,0)
	;	
\end{circuitikz}
 & C-Typ & C-Klemmenkreis & C-Klemmentrennbündel\\
& & Nullstelle bei $\omega \rightarrow \infty$ & Pol bei $\omega \rightarrow
0$\\
\begin{circuitikz}[scale=2, european, american inductors, yscale=0.3]
\ctikzset{bipoles/length=1.2cm}
\draw (0,0)
	to[C, *-] (1,0)
	to[L, -*] (2,0)
	;	
\end{circuitikz}
 & S-Typ & C-Klemmentrennbündel &
L-Klemmentrennbündel\\
\tikzexternaldisable
\begin{circuitikz}[scale=2, european, american inductors, yscale=0.3]
\ctikzset{bipoles/length=0.5cm}
\draw (0,0)
	to[short, *-] (0.1cm,0)
	(0.1cm,0.5) to[short] (0.1cm,-0.5)
	(0.1cm,0.5) to [L] (0.6cm,0.5)
	(0.1cm,-0.5) to [C] (0.6cm,-0.5)
	(0.6cm,0.5) to [short] (0.6cm,-0.5)
	(0.6cm,0) to [short, -*] (0.7cm,0)
	;	
\end{circuitikz}
\tikzexternalenable & P-Typ & C-Klemmenkreis & L-Klemmenkreis\\
\end{tabular}

\begin{figure}[!h]
\centering
\subfloat[C-Klemmentrennbündel]{
	\begin{circuitikz}[scale=2, european, american inductors]
\ctikzset{bipoles/length=1.2cm}
\draw
	(0,0) to [L, *-] (1,0)
	to [C, color=red] (2,0)
	to [C] (3,0)
	to [C] (3,-1)
	to [L] (2,-1)
	to [L] (1,-1)
	to [L, -*] (0,-1)
	(1,0) to [C, color=red] (1,-1)
	(2,-1) to [C] (2,0)
	to [L] (3,-1)
	(0,0) to [C, color=red] (0,-1)
	;
\end{circuitikz}

	\label{fig:ret:klemmenbuendel}
}
\qquad
\subfloat[C-Klemmenkreis]{
	\begin{circuitikz}[scale=2, european, american inductors]
\ctikzset{bipoles/length=1.2cm}
\draw[color=red]
	(0,0)
	to [C, *-] (1,0)
	to [C] (2,0)
	to [C] (2,-1)
	to [C] (1,-1)
	to [C, -*] (0,-1)
	;
\draw
	(2,0)
	to [C] (3,0)
	to [L] (3,-1)
	to [C] (2,-1)
	(1,0) to [L] (1,-1)
	(2,0) to [L] (3,-1)
	(0,0) to [L] (0,-1)
	;
\end{circuitikz}

	\label{fig:ret:klemmenkreis} 
}
\caption{C-Klemmentrennbündel und Klemmenkreis}
\label{fig:ret}
\end{figure}





\subsection{RET-Synthese}
\index{RET!Synthese}
Für die Reaktanzeintor-Synthese gibt es vier systematische Verfahren:

\begin{tikzpicture}[level distance=50mm,
level 1/.style={sibling distance=15mm},
level 2/.style={sibling distance=10mm}]
	\node {Es gibt 4 systematische Verfahren}[grow=right]
		child {node {Partialbruchzerlegung}
			child {node {nach Polen der Impedanz}}
			child {node {nach Polen der Admittanz}}
		}
		child {node {Kettenbruchzerlegung}
			child {node {1. Art}}
			child {node {2. Art}}
		};
\end{tikzpicture}\\

\textbf{geg:} RET-Funktion\\
\textbf{ges:} zugehörige Schaltung\\


\subsubsection{Synthese mittels Partialbruchzerlegung}
\index{RET!Synthese!Partialbruchzerlegung}
\paragraph{a) Zerlegung nach Polen der Impedanz}
Beispiel: $\underline{Z}(p) = \frac{p^4+4p^2+3}{2p^5+12p^3+16p}$\\
\begin{tabular}{lll}
	Vorabklärung & $p=0$ & Pol \\
	& $p \rightarrow \infty$ & Nullstelle \\
	& 5 Elemente & wegen $p^5$ \\
\end{tabular}
\begin{figure}[!h]
	\centering
	\usepgflibrary{shapes.misc}

\begin{tikzpicture}[domain=0:4, smooth]
% Achsen
\draw[->, thick] (-1.7,0) -- +(9.5,0) node[right] {$\omega$}; % Horizontal
\draw[->, thick] (-1.57,-3.2) -- +(0,6.4) node[above] {X}; % Vertikal

% Plots
\draw[color=green!70!black, thick] plot[domain=-1.27:1.27] (\x,{tan(\x r)}) node[right] {}; % Erster Tan
\draw[color=green!70!black, thick] plot[domain=1.87:4.41] (\x,{tan(\x r)}) node[right] {}; % zweiter Tan
\draw[color=green!70!black, thick] plot[domain=4.91:8] (\x,{-1/(\x - 4.6)}) node[right] {}; % letzte kurve

% Poolstellen
\draw[dashed, thick, draw=red] (1.57,-3.2) -- +(0,6.4); % Poolstelle 1
\draw[dashed, thick, draw=red] (4.71,-3.2) -- +(0,6.4); % Poolstelle 2

\node[cross out, draw=red, thick] at (-1.57,0) {};

\node[cross out, draw=red, thick] (wr1) at (1.57,0) {};
\node at(2,-0.3) {$\omega_{r1}$};

\node[cross out, draw=red, thick] at (4.71,0) {};
\node at (5.2,-0.3) {$\omega_{r2}$};

% Nullstellen
\node[rounded rectangle, draw=blue, thick] at(0,0) {};
\node[rounded rectangle, draw=blue, thick] at(3.141,0) {};
\node[rounded rectangle, draw=blue, thick] at(8.7,0) {};


\end{tikzpicture}

	\caption{Pol-/Nullstellen Diagramm}
	\label{fig:RetPolNullstelle}
\end{figure}\\
In der Grafik sieht man dass es sich bei $\omega \rightarrow 0$ und $\omega
\rightarrow \infty$ wie ein C verhält. Draus folgt dass es ein C-Typ ist.\\
\begin{figure}[!h]
	\centering
	\tikzexternaldisable
\begin{circuitikz}[scale=2, european, american inductors, yscale=0.8]
\ctikzset{bipoles/length=1.2cm}
\draw (0,0)
	to[short, *-] (0,0)
	to[C] (1,0)
	(1,0.5) to [short] (1,-0.5)
	(1,0.5) to [L] (2,0.5)
	(1,-0.5) to [C] (2,-0.5)
	(2,0.5) to [short] (2,-0.5)
	(2,0) to [short] (2.5,0)
	(2.5,0.5) to [short] (2.5,-0.5)
	(2.5,0.5) to [L=$L_K$] (3.5,0.5)
	(2.5,-0.5) to [C,l_=$C_K$] (3.5,-0.5)
	(3.5,0.5) to [short] (3.5,-0.5)
	(3.5,0) to [short, -*] (4,0)
	(1.5,0) node{$\omega_{r1}$}
	(3,0) node{$\omega_{r2}$}
	;	
\end{circuitikz}
\tikzexternalenable

	\caption{Reaktanzeintor Beispiel 1}
	\label{fig:RetSyntheseBsp1S}
\end{figure}
\begin{align}
\underline{Z}_K&=\frac{1}{pC_k+\frac{1}{pL_k}} =
\frac{pL_k}{p^2L_KC_K+1}\nonumber\\
\underline{Z}(p)&=\frac{p^4+4p^2+3}{2p^5+12p^3+16p}\nonumber
\end{align}
Nennergrad $>$ Zählergrad: OK\\
\begin{align}
\underline{N}(p)&=2p^5+12p^3+16p=2p\left(p^4+6p^2+8\right)=2p\left(p^2+4\right)\left(p^2+2\right)\nonumber
\end{align}
Nullstellen vom Nenner bei: $p=0,\ p^2=-4,\ p^2=-2$\\
\begin{align}
\underline{Z}_K&=\frac{A}{p}+\frac{Bp}{p^2+4}+\frac{Cp}{p^2+2}\nonumber
\end{align}
$\rightarrow A,B,C$ bestimmen, Gleichung mit ganzem Nenner multiplizieren\\
\begin{align}
p^4+4p^2+3&=2A(p^2+4)(p^2+2)+Bp\cdot 2p(p^2+2)+Cp(2p)(p^2+4)\nonumber\\
p&=0:\ 3=2A\cdot 4 \cdot 2 &\Rightarrow A=\frac{3}{16}\nonumber\\
p^2&=-4:\ \underbrace{16-16+3}_{3}=B\cdot \underbrace{2\cdot (-4)(-2)}_{16}
&\Rightarrow B=\frac{3}{16}\nonumber\\ 
p^2&=-2:\ ... &\Rightarrow C=\frac{1}{8}\nonumber\\
\underline{Z}(p)&=\frac{\frac{3}{16}}{p}+\frac{\frac{3}{16}p}{p^2+4}+\frac{\frac{1}{8}p}{p^2+2}\nonumber\\
&=\frac{1}{\frac{16}{3}p}+\frac{1}{\frac{16}{3}p+\frac{4\cdot\frac{16}{3}}{p}}+\frac{1}{8p+\frac{2\cdot8}{p}}\nonumber\\
&=\frac{1}{\frac{16}{3}p}+\frac{1}{\frac{16}{3}p+\frac{1}{\frac{3}{64}p}}+\frac{1}{8p+\frac{1}{\frac{1}{16}p}}\nonumber
\end{align}
\begin{figure}[!h]
	\centering
	\tikzexternaldisable
\begin{circuitikz}[scale=2, european, american inductors, yscale=0.8]
\ctikzset{bipoles/length=1.2cm}
\draw (0,0)
	to[short, *-] (0,0)
	to[C=$\frac{16}{3}F$] (1,0)
	(1,0.5) to [short] (1,-0.5)
	(1,0.5) to [L=$\frac{3}{64}H$] (2,0.5)
	(1,-0.5) to [C=$\frac{16}{3}F$] (2,-0.5)
	(2,0.5) to [short] (2,-0.5)
	(2,0) to [short] (2.5,0)
	(2.5,0.5) to [short] (2.5,-0.5)
	(2.5,0.5) to [L=$\frac{1}{16}H$] (3.5,0.5)
	(2.5,-0.5) to [C=$8F$] (3.5,-0.5)
	(3.5,0.5) to [short] (3.5,-0.5)
	(3.5,0) to [short, -*] (4,0);	
\end{circuitikz}
\tikzexternalenable
	\caption{Lösung Reaktanzeintor Beispiel 1}
	\label{fig:RetSyntheseBsp1SL}
\end{figure}\\


\paragraph{b) Zerlegung nach Polen der Admittanz}
\begin{figure}[!ht]
	\centering
	\tikzexternaldisable
\begin{circuitikz}[scale=2, european, american inductors, yscale=0.8]
\ctikzset{bipoles/length=1.2cm}
\draw (0,0)
	to[short, *-] (3,0)
	to[C] (3,-1)
	to[L] (3,-2)
	to[short, -*] (0,-2)
	(1,0) to [C] (1,-2)
	(2,0)	to [C] (2,-1)
	to [L]	(2,-2);
\end{circuitikz}
\tikzexternalenable
	\caption{Reaktanzeintor Beispiel 2}
	\label{fig:RetSyntheseBsp2}
\end{figure}

Pole der Admittanz $\rightarrow \infty$ grosse Leitfähigkeit (bei
Resonanzfrequenz)\\
$$\underline{Y}(p)=\frac{1}{\underline{Z}(p)}=\frac{2p^5+12p^3+16p}{p^4+4p^2+3}$$\\
Zählergrad $>$ Nennergrad: zuerst ausdividieren, Rest ist eine echt gebrochen
rationale Funktion.
\begin{align}
\underline{Y}(p)&=(2p^5+12p^3+16p):(p^4+4p^2+3)=2p+\frac{4p^3+10p}{p^4+4p^2+3}\nonumber
\end{align}
\begin{align}
\underline{Y}_{rest}(p)=\frac{4p^3+10p}{p^4+4p^2+3}&=\frac{4p^3+10p}{(p^2+3)(p^2+1)}=\frac{Ap}{p^2+3}+\frac{Bp}{p^2+1}\nonumber\\
4p^3+10p&=Ap(p^2+1)+Bp(p^2+3)\nonumber\\
4p^2+10&=A(p^2+1)+B(p^2+3)\nonumber
\end{align}
\begin{align}
p^2&=-3:\ \underbrace{-12+10}_{-2}=A(-2) &\Rightarrow A=1\nonumber\\
p^2&=-1:\ \underbrace{-4+10}_{6}=B(2) &\Rightarrow B=3\nonumber\\
\underline{Y}(p)&=2p+\frac{p}{p^2+3}+\frac{3p}{p^2+1}=2p+\frac{1}{p+\frac{1}{\frac{1}{3}p}}+\frac{1}{\frac{1}{3}p+\frac{1}{3p}}\nonumber
\end{align}
\begin{figure}[!ht]
	\centering
	\begin{circuitikz}[scale=2, european, american inductors, yscale=0.8]
\ctikzset{bipoles/length=1.2cm}
\draw (0,0)
	to[short, *-] (3,0)
	to[C=$3F$] (3,-1)
	to[L=$\frac{1}{3}H$] (3,-2)
	to[short, -*] (0,-2)
	(1,0) to [C=$2F$] (1,-2)
	(2,0)	to [C=$\frac{1}{3}F$] (2,-1)
	to [L=$1H$]	(2,-2);
\end{circuitikz}
	\caption{Lösung Reaktanzeintor Beispiel 2}
	\label{fig:RetSyntheseBsp2L}
\end{figure}\\


\subsubsection{Synthese mittels Kettenbruchzerlegung}
\index{RET!Synthese!Kettenbruchzerlegung}
\paragraph{1. Art}
Man geht aus von unecht gebrochen rationaler Funktion und
spaltet Pol bei $\infty$ ab. Dann invertiert man, etc.
\begin{align}
	\underline{Y}(p)&=\left(2p^5+12p^3+16p\right):\left(p^4+4p^2+3\right)=2p+\frac{4p^3+10p}{p^4+4p^2+3},
	&\underline{Y}_1=\frac{4p^3+10p}{p^4+4p^2+3}\nonumber\\
	\underline{Z}_1&=\left(p^4+4p^2+3\right):\left(4p^3+10p\right)=\frac{1}{4}p+\frac{\frac{3}{2}p^2+3}{4p^3+10p},
	&\underline{Z}_2=\frac{\frac{3}{2}p^2+3}{4p^3+10p}\nonumber\\
	\underline{Y}_2&=\left(4p^3+10p\right):\left(\frac{3}{2}p^2+3\right)=\frac{8}{3}p+\frac{2p}{\frac{3}{2}p^2+3},
	&\underline{Z}_3=\frac{\frac{3}{2}p^2+3}{4p^3+10p}\nonumber\\
	\underline{Z}_3&=\left(\frac{3}{2}p^2+3\right):\left(2p\right)=\frac{3}{4}p+\frac{3}{2p}=\frac{3}{4}p+\frac{1}{\frac{2}{3}p}\nonumber\\
	\frac{3}{4}&=L\nonumber\\ \frac{2}{3}&=C\nonumber
\end{align}