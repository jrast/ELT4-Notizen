\section{Reaktanz Eintore}
\index{Reaktanz Eintore|see{RET}}
\index{RET}
Ein Reaktanz Eintor ist eine Schaltung aus lauter $L$ und $C$ mit genau zwei
Anschlüssen.

Skript von Herrn Neubauer, Seiten 27ff

Bei $\omega = 0$ und $\omega \rightarrow \infty$ hat $X(\omega)$ entweder einen
Pol oder eine Nullstelle.

\subsection{RET-Eigenschaften}
\index{RET!Eigenschaften}
Reaktanz-Eintore sind aufgrund ihrer Eintor- bzw. Impedanzfunktionen
$\underline{Z}(j\omega)$ charakteriesiert deren besondere
Eigenschaften sich wie folgt zusammenstellen lassen.

\begin{description}
[style=multiline,topsep=0pt,leftmargin=4.5cm,rightmargin=2cm]
  \item[Rationale Funktionen] Zähler und Nenner immer reell und positiv.
\end{description}
\begin{align}
  \underline{Z}(j \omega) 
  &= \frac{
       a_n \left( j\omega \right)^n + 
       a_{n-2} \left( j\omega \right)^{n-2} + 
       a_{n-4} \left( j\omega \right)^{n-4} + 
       \ldots
     }
     {
       b_m \left( j\omega \right)^m + 
       b_{m-2} \left( j\omega \right)^{m-2} +
       b_{m-4} \left( j\omega \right)^{m-4} + 
       \ldots
     } \\
  &= K \frac
     {
       \left[\left( j\omega \right)^2 + j \omega_1^2 \right]
       \left[\left( j\omega \right)^2 + j \omega_3^2 \right]
       \left[\left( j\omega \right)^2 + j \omega_5^2 \right]
       \ldots
     }
     {
       \left[\left(j\omega\right)^2 + j \omega_2^2 \right]
       \left[\left(j\omega\right)^2 + j \omega_4^2 \right]
       \left[\left(j\omega\right)^2 + j \omega_6^2 \right]
       \ldots
     }
  = \frac
    {
      P_n\left( j\omega \right)
    }
    {
      Q_m \left( j\omega \right)
    }
\end{align}
\begin{description}
[style=multiline,topsep=0pt,leftmargin=4.5cm,rightmargin=2cm]
  \item[Hohe Frequenzen] ($j \omega \rightarrow \infty$) höchste Potenz
  entscheidend:
    \begin{description}
    [style=multiline,topsep=0pt,leftmargin=2.5cm,rightmargin=0cm]
      \item[n grösser m] Polstelle $\longrightarrow$ $\frac{a_n}{b_m} = L_\infty$
      \item[n kleiner m] Nullstelle $\longrightarrow$ $\frac{a_n}{b_m} =
        \frac{1}{C_\infty}$
    \end{description}
  \item[Tiefe Frequenzen] ($j \omega \rightarrow 0$) tiefste Potenz
    entscheidend:
    \begin{description}
    [style=multiline,topsep=0pt,leftmargin=2.5cm,rightmargin=0cm]
      \item[n ungerade] ($\frac{\ldots + a_3(j \omega)^3 + a_1 (j\omega)}{\ldots
        + b_2(j\omega)^2 + b_0}$) Nullstelle für $j\omega \rightarrow 0$ \newline
        $\frac{a_1}{b} = L_0$ für ganz tiefe Frequenzen 
      \item[n gerade] ($\frac{\ldots + a_2(j \omega)^2 + a_0}{\ldots
        + b_3(j\omega)^3 + b_1(j\omega)}$) \newline
        $\frac{a_0}{b_1} = \frac{1}{C_0}$ für ganz tiefe Frequenzen
    \end{description}
\end{description}



\subsection{RET-Typen}
\index{RET!Typen}
Man unterscheidet die Reaktanzeintore in vier Typen. Dazu betrachtet man das
Verhalten für $\omega \rightarrow 0$ und $\omega \rightarrow \infty$. Alternativ
kann auch die Schaltung analysiert werden indem man Klemmentrennbündel und
Klemmenkreise bildet (siehe Abbildung \ref{fig:ret}). Anhand dieser
Informationen kann aus der Tabelle \ref{tab:RETTyp} der Typ bestimmt werden.

\begin{sidewaystable}
\begin{tabular}{|l|l|l|l|l|l|l|l|l|l|l|}
\hline
	\textbf{Symbol} &
	\textbf{Typ} &
	\textbf{Reaktanz} &
	\multicolumn{2}{|c|}{\textbf{Impedanzfunktion}} &
	\multicolumn{2}{|c|}{\textbf{Eigenschaften}} &
	$\bf \boldsymbol\omega = 0$ &
	$\bf \boldsymbol\omega \rightarrow \boldsymbol\infty$ &
	n & 
	m\\
	& & & Summenform & Produktform & & & & & & \\
\hline
	\parbox[c][1.5cm]{1.2cm}{\begin{circuitikz}[scale=2, european, american inductors, yscale=0.4]
\ctikzset{bipoles/length=1.2cm}
\draw (0,0)
	to[L, *-*] (2,0)
	;	
\end{circuitikz}
} &
	L-Typ &
	Bild Jürg &
	$ \underline{Z}(p)=p\frac{a_np^{n-1}+ \ldots +a_1}{b_mp^m+ \ldots b_0}$&
	$=\frac{j\omega
	L_{\infty}[(j\omega)^2+\omega_3^2][\ldots]}{[(j\omega)^2+\omega_2^2][\ldots]}$&
	L-Kreis &
	L-TB &
	Null &
	Pol &
	ug &
	$m=n-1$\\
\hline
	\parbox[c][1.5cm]{1.2cm}{\begin{circuitikz}[scale=2, european, american inductors, yscale=0.4]
\ctikzset{bipoles/length=1.2cm}
\draw (0,0)
	to[C, *-*] (2,0)
	;	
\end{circuitikz}
} &
	C-Typ &
	Bild Jürg &
	$\underline{Z}(p)=\frac{1}{p}\frac{a_np^{n}+ \ldots +a_0}{b_mp^{m-1}+\ldots	b_1}$&
	$=\frac{[(j\omega)^2+\omega_2^2][\ldots]}{j\omega C_{\infty}[(j\omega)^2+\omega_3^2][\ldots]}$ &
	C-Kreis &	
	C-TB &
	Pol &
	Null &
	g &
	$m=n+1$ \\
\hline
	\parbox[c][1.5cm]{1.2cm}{\begin{circuitikz}[scale=2, european, american inductors, yscale=0.3]
\ctikzset{bipoles/length=1.2cm}
\draw (0,0)
	to[C, *-] (1,0)
	to[L, -*] (2,0)
	;	
\end{circuitikz}
} &
	S-Typ &
	Bild Jürg &
	$\underline{Z}(p)=\frac{1}{p}\frac{a_np^{n}+\ldots+a_0}{b_mp^{m-1}+\ldots
	b_1}$&
	$=\frac{[L_{\infty}(j\omega)^2+\omega_2^2][(j\omega)^2+\omega_4^2][\ldots]}{j\omega[(j\omega)^2+\omega_3^2]\ldots]}$ &
	C-TB &
	L-TB &
	Pol &
	Pol &
	g &
	$m=n-1$ \\
\hline
	\parbox[c][1.5cm]{1.2cm}{\tikzexternaldisable
\begin{circuitikz}[scale=2, european, american inductors, yscale=0.3]
\ctikzset{bipoles/length=0.5cm}
\draw (0,0)
	to[short, *-] (0.1cm,0)
	(0.1cm,0.5) to[short] (0.1cm,-0.5)
	(0.1cm,0.5) to [L] (0.6cm,0.5)
	(0.1cm,-0.5) to [C] (0.6cm,-0.5)
	(0.6cm,0.5) to [short] (0.6cm,-0.5)
	(0.6cm,0) to [short, -*] (0.7cm,0)
	;	
\end{circuitikz}
\tikzexternalenable} &
	P-Typ &
	Bild Jürg &
	$\underline{Z}(p)=p\frac{a_np^{n-1}+\ldots +a_1}{b_mp^m+\ldots b_0}$&
	$=\frac{j\omega[(j\omega)^2+\omega_3^2][\ldots]}{C_{\infty}[(j\omega)^2+\omega_2^2][(j\omega)^2+\omega_4^2]\ldots]}$&
	C-Kreis & L-Kreis &
	Null &
	Null &
	ug &
	$m=n+1$ \\
\hline
\end{tabular}
\caption[Bestimmung des RET-Typ]{Bestimmung des RET-Typ. Die Bezeichnungen
Klemmentrennbündel und Klemmenkreis wurden abgekürzt zu TB und Kreis. Gerade
wurde zu g und ungerade zu ug abgekürzt.}
\label{tab:RETTyp}
\end{sidewaystable}

\begin{figure}[!ht]
\centering
\subfloat[C-Klemmentrennbündel]{
	\begin{circuitikz}[scale=2, european, american inductors]
\ctikzset{bipoles/length=1.2cm}
\draw
	(0,0) to [L, *-] (1,0)
	to [C, color=red] (2,0)
	to [C] (3,0)
	to [C] (3,-1)
	to [L] (2,-1)
	to [L] (1,-1)
	to [L, -*] (0,-1)
	(1,0) to [C, color=red] (1,-1)
	(2,-1) to [C] (2,0)
	to [L] (3,-1)
	(0,0) to [C, color=red] (0,-1)
	;
\end{circuitikz}

	\label{fig:ret:klemmenbuendel}
}
\qquad
\subfloat[C-Klemmenkreis]{
	\begin{circuitikz}[scale=2, european, american inductors]
\ctikzset{bipoles/length=1.2cm}
\draw[color=red]
	(0,0)
	to [C, *-] (1,0)
	to [C] (2,0)
	to [C] (2,-1)
	to [C] (1,-1)
	to [C, -*] (0,-1)
	;
\draw
	(2,0)
	to [C] (3,0)
	to [L] (3,-1)
	to [C] (2,-1)
	(1,0) to [L] (1,-1)
	(2,0) to [L] (3,-1)
	(0,0) to [L] (0,-1)
	;
\end{circuitikz}

	\label{fig:ret:klemmenkreis} 
}
\caption[Klemmentrennbündel und Klemmenkreis]{C-Klemmentrennbündel und
C-Klemmenkreis, gilt analog für Induktivitäten}
\label{fig:ret}
\end{figure}




\subsection{RET-Synthese}
\index{RET!Synthese}
Für die Reaktanzeintor-Synthese gibt es vier systematische Verfahren:

\begin{figure}[H]
\begin{center}
  \begin{tikzpicture}[level distance=50mm,
level 1/.style={sibling distance=15mm},
level 2/.style={sibling distance=10mm}]
	\node {Es gibt 4 systematische Verfahren}[grow=right]
		child {node {Partialbruchzerlegung}
			child {node {nach Polen der Impedanz}}
			child {node {nach Polen der Admittanz}}
		}
		child {node {Kettenbruchzerlegung}
			child {node {1. Art}}
			child {node {2. Art}}
		};
\end{tikzpicture}\\
\end{center}
\end{figure}

\textbf{geg:} RET-Funktion\\
\textbf{ges:} zugehörige Schaltung\\


\subsubsection{Synthese mittels Partialbruchzerlegung}
\index{RET!Synthese!Partialbruchzerlegung}
\paragraph{a) Zerlegung nach Polen der Impedanz}
Beispiel: $\underline{Z}(p) = \frac{p^4+4p^2+3}{2p^5+12p^3+16p}$\\
\begin{tabular}{lll}
	Vorabklärung & $p=0$ & Pol \\
	& $p \rightarrow \infty$ & Nullstelle \\
	& 5 Elemente & wegen $p^5$ \\
\end{tabular}
\begin{figure}[!h]
	\centering
	\usepgflibrary{shapes.misc}

\begin{tikzpicture}[domain=0:4, smooth]
% Achsen
\draw[->, thick] (-1.7,0) -- +(9.5,0) node[right] {$\omega$}; % Horizontal
\draw[->, thick] (-1.57,-3.2) -- +(0,6.4) node[above] {X}; % Vertikal

% Plots
\draw[color=green!70!black, thick] plot[domain=-1.27:1.27] (\x,{tan(\x r)}) node[right] {}; % Erster Tan
\draw[color=green!70!black, thick] plot[domain=1.87:4.41] (\x,{tan(\x r)}) node[right] {}; % zweiter Tan
\draw[color=green!70!black, thick] plot[domain=4.91:8] (\x,{-1/(\x - 4.6)}) node[right] {}; % letzte kurve

% Poolstellen
\draw[dashed, thick, draw=red] (1.57,-3.2) -- +(0,6.4); % Poolstelle 1
\draw[dashed, thick, draw=red] (4.71,-3.2) -- +(0,6.4); % Poolstelle 2

\node[cross out, draw=red, thick] at (-1.57,0) {};

\node[cross out, draw=red, thick] (wr1) at (1.57,0) {};
\node at(2,-0.3) {$\omega_{r1}$};

\node[cross out, draw=red, thick] at (4.71,0) {};
\node at (5.2,-0.3) {$\omega_{r2}$};

% Nullstellen
\node[rounded rectangle, draw=blue, thick] at(0,0) {};
\node[rounded rectangle, draw=blue, thick] at(3.141,0) {};
\node[rounded rectangle, draw=blue, thick] at(8.7,0) {};


\end{tikzpicture}

	\caption{Pol-/Nullstellen Diagramm}
	\label{fig:RetPolNullstelle}
\end{figure}\\
In der Grafik sieht man dass es sich bei $\omega \rightarrow 0$ und $\omega
\rightarrow \infty$ wie ein C verhält. Draus folgt dass es ein C-Typ ist.\\
\begin{figure}[!h]
	\centering
	\tikzexternaldisable
\begin{circuitikz}[scale=2, european, american inductors, yscale=0.8]
\ctikzset{bipoles/length=1.2cm}
\draw (0,0)
	to[short, *-] (0,0)
	to[C] (1,0)
	(1,0.5) to [short] (1,-0.5)
	(1,0.5) to [L] (2,0.5)
	(1,-0.5) to [C] (2,-0.5)
	(2,0.5) to [short] (2,-0.5)
	(2,0) to [short] (2.5,0)
	(2.5,0.5) to [short] (2.5,-0.5)
	(2.5,0.5) to [L=$L_K$] (3.5,0.5)
	(2.5,-0.5) to [C,l_=$C_K$] (3.5,-0.5)
	(3.5,0.5) to [short] (3.5,-0.5)
	(3.5,0) to [short, -*] (4,0)
	(1.5,0) node{$\omega_{r1}$}
	(3,0) node{$\omega_{r2}$}
	;	
\end{circuitikz}
\tikzexternalenable

	\caption{Reaktanzeintor Beispiel 1}
	\label{fig:RetSyntheseBsp1S}
\end{figure}
\begin{align}
\underline{Z}_K&=\frac{1}{pC_k+\frac{1}{pL_k}} =
\frac{pL_k}{p^2L_KC_K+1}\nonumber\\
\underline{Z}(p)&=\frac{p^4+4p^2+3}{2p^5+12p^3+16p}\nonumber
\end{align}
Nennergrad $>$ Zählergrad: OK\\
\begin{align}
\underline{N}(p)&=2p^5+12p^3+16p=2p\left(p^4+6p^2+8\right)=2p\left(p^2+4\right)\left(p^2+2\right)\nonumber
\end{align}
Nullstellen vom Nenner bei: $p=0,\ p^2=-4,\ p^2=-2$\\
\begin{align}
\underline{Z}_K&=\frac{A}{p}+\frac{Bp}{p^2+4}+\frac{Cp}{p^2+2}\nonumber
\end{align}
$\rightarrow A,B,C$ bestimmen, Gleichung mit ganzem Nenner multiplizieren\\
\begin{align}
p^4+4p^2+3&=2A(p^2+4)(p^2+2)+Bp\cdot 2p(p^2+2)+Cp(2p)(p^2+4)\nonumber\\
p&=0:\ 3=2A\cdot 4 \cdot 2 &\Rightarrow A=\frac{3}{16}\nonumber\\
p^2&=-4:\ \underbrace{16-16+3}_{3}=B\cdot \underbrace{2\cdot (-4)(-2)}_{16}
&\Rightarrow B=\frac{3}{16}\nonumber\\ 
p^2&=-2:\ ... &\Rightarrow C=\frac{1}{8}\nonumber\\
\underline{Z}(p)&=\frac{\frac{3}{16}}{p}+\frac{\frac{3}{16}p}{p^2+4}+\frac{\frac{1}{8}p}{p^2+2}\nonumber\\
&=\frac{1}{\frac{16}{3}p}+\frac{1}{\frac{16}{3}p+\frac{4\cdot\frac{16}{3}}{p}}+\frac{1}{8p+\frac{2\cdot8}{p}}\nonumber\\
&=\frac{1}{\frac{16}{3}p}+\frac{1}{\frac{16}{3}p+\frac{1}{\frac{3}{64}p}}+\frac{1}{8p+\frac{1}{\frac{1}{16}p}}\nonumber
\end{align}
\begin{figure}[!h]
	\centering
	\tikzexternaldisable
\begin{circuitikz}[scale=2, european, american inductors, yscale=0.8]
\ctikzset{bipoles/length=1.2cm}
\draw (0,0)
	to[short, *-] (0,0)
	to[C=$\frac{16}{3}F$] (1,0)
	(1,0.5) to [short] (1,-0.5)
	(1,0.5) to [L=$\frac{3}{64}H$] (2,0.5)
	(1,-0.5) to [C=$\frac{16}{3}F$] (2,-0.5)
	(2,0.5) to [short] (2,-0.5)
	(2,0) to [short] (2.5,0)
	(2.5,0.5) to [short] (2.5,-0.5)
	(2.5,0.5) to [L=$\frac{1}{16}H$] (3.5,0.5)
	(2.5,-0.5) to [C=$8F$] (3.5,-0.5)
	(3.5,0.5) to [short] (3.5,-0.5)
	(3.5,0) to [short, -*] (4,0);	
\end{circuitikz}
\tikzexternalenable
	\caption{Lösung Reaktanzeintor Beispiel 1}
	\label{fig:RetSyntheseBsp1SL}
\end{figure}\\


\paragraph{b) Zerlegung nach Polen der Admittanz}
\begin{figure}[!ht]
	\centering
	\tikzexternaldisable
\begin{circuitikz}[scale=2, european, american inductors, yscale=0.8]
\ctikzset{bipoles/length=1.2cm}
\draw (0,0)
	to[short, *-] (3,0)
	to[C] (3,-1)
	to[L] (3,-2)
	to[short, -*] (0,-2)
	(1,0) to [C] (1,-2)
	(2,0)	to [C] (2,-1)
	to [L]	(2,-2);
\end{circuitikz}
\tikzexternalenable
	\caption{Reaktanzeintor Beispiel 2}
	\label{fig:RetSyntheseBsp2}
\end{figure}

Pole der Admittanz $\rightarrow \infty$ grosse Leitfähigkeit (bei
Resonanzfrequenz)\\
$$\underline{Y}(p)=\frac{1}{\underline{Z}(p)}=\frac{2p^5+12p^3+16p}{p^4+4p^2+3}$$\\
Zählergrad $>$ Nennergrad: zuerst ausdividieren, Rest ist eine echt gebrochen
rationale Funktion.
\begin{align}
\underline{Y}(p)&=(2p^5+12p^3+16p):(p^4+4p^2+3)=2p+\frac{4p^3+10p}{p^4+4p^2+3}\nonumber
\end{align}
\begin{align}
\underline{Y}_{rest}(p)=\frac{4p^3+10p}{p^4+4p^2+3}&=\frac{4p^3+10p}{(p^2+3)(p^2+1)}=\frac{Ap}{p^2+3}+\frac{Bp}{p^2+1}\nonumber\\
4p^3+10p&=Ap(p^2+1)+Bp(p^2+3)\nonumber\\
4p^2+10&=A(p^2+1)+B(p^2+3)\nonumber
\end{align}
\begin{align}
p^2&=-3:\ \underbrace{-12+10}_{-2}=A(-2) &\Rightarrow A=1\nonumber\\
p^2&=-1:\ \underbrace{-4+10}_{6}=B(2) &\Rightarrow B=3\nonumber\\
\underline{Y}(p)&=2p+\frac{p}{p^2+3}+\frac{3p}{p^2+1}=2p+\frac{1}{p+\frac{1}{\frac{1}{3}p}}+\frac{1}{\frac{1}{3}p+\frac{1}{3p}}\nonumber
\end{align}
\begin{figure}[!ht]
	\centering
	\begin{circuitikz}[scale=2, european, american inductors, yscale=0.8]
\ctikzset{bipoles/length=1.2cm}
\draw (0,0)
	to[short, *-] (3,0)
	to[C=$3F$] (3,-1)
	to[L=$\frac{1}{3}H$] (3,-2)
	to[short, -*] (0,-2)
	(1,0) to [C=$2F$] (1,-2)
	(2,0)	to [C=$\frac{1}{3}F$] (2,-1)
	to [L=$1H$]	(2,-2);
\end{circuitikz}
	\caption{Lösung Reaktanzeintor Beispiel 2}
	\label{fig:RetSyntheseBsp2L}
\end{figure}\\


\subsubsection{Synthese mittels Kettenbruchzerlegung}
\index{RET!Synthese!Kettenbruchzerlegung}
\paragraph{1. Art}
Man geht aus von unecht gebrochen rationaler Funktion und
spaltet einen Pol bei $\infty$ ab. Dann invertiert man, etc.
\begin{align}
	\underline{Y}(p)&=\left(2p^5+12p^3+16p\right):\left(p^4+4p^2+3\right)=2p+\frac{4p^3+10p}{p^4+4p^2+3},
	&\underline{Y}_1=\frac{4p^3+10p}{p^4+4p^2+3}\nonumber\\
	\underline{Z}_1&=\left(p^4+4p^2+3\right):\left(4p^3+10p\right)=\frac{1}{4}p+\frac{\frac{3}{2}p^2+3}{4p^3+10p},
	&\underline{Z}_2=\frac{\frac{3}{2}p^2+3}{4p^3+10p}\nonumber\\
	\underline{Y}_2&=\left(4p^3+10p\right):\left(\frac{3}{2}p^2+3\right)=\frac{8}{3}p+\frac{2p}{\frac{3}{2}p^2+3},
	&\underline{Z}_3=\frac{\frac{3}{2}p^2+3}{4p^3+10p}\nonumber\\
	\underline{Z}_3&=\left(\frac{3}{2}p^2+3\right):\left(2p\right)=\frac{3}{4}p+\frac{3}{2p}=\frac{3}{4}p+\frac{1}{\frac{2}{3}p}\nonumber\\
	\frac{3}{4}&=L\nonumber\\
	\frac{2}{3}&=C\nonumber\\
	\underline{Y}(p)&=2p+\frac{1}{\frac{1}{4}p+\frac{1}{\frac{8}{3}p+\frac{1}{\frac{3}{4}p+\frac{1}{\frac{2}{3}p}}}}\nonumber
\end{align}
\begin{circuitikz}[scale=4, european, american inductors, yscale=0.8]
\ctikzset{bipoles/length=1.2cm}
\draw (0,0)
	to[short, *-] (1,0)
	to[L=$\frac{1}{4}H$] (2,0)
	to[L=$\frac{3}{4}H$] (3,0)
	to[C=$\frac{2}{3}F$] (3,-1)
	to[short, -*] (0,-1)
	(1,0) to [C=$2F$] (1,-1)
	(2,0) to [C=$\frac{8}{3}F$] (2,-1)
	;
\draw[dashed, very thick, color=green]
	(0.5,-1.2) to[short] (0.5,-0.5)
	to[short] (0.8,-0.5)
	(1.3,-1.2) to[short] (1.3,-0.5)
	to[short] (1.5,-0.5)
	(2.3,-1.2) to[short] (2.3,-0.5)
	to[short] (2.5,-0.5)
	;
\draw
	(0.5,-1.3) node[draw=none, fill=none, color=green] {$\underline{Y}(p)$}
	(1.3,-1.3) node[draw=none, fill=none, color=green] {$\underline{Z}_1(p)$}
	(2.3,-1.3) node[draw=none, fill=none, color=green] {$\underline{Z}_3(p)$}
;
\end{circuitikz}

\paragraph{2. Art}
Es werden Pole bei $p=0$ abgespalten.\\
(Ähnlich wie Kettenbruchzerlegung 1. Art,
aber man beginnt bei tiefsten Potenzen)


\begin{align}
	\underline{Y}(p)&=\frac{16p+12p^3+2p^5}{3+4p^2+p^4}\rightarrow\text{Kein
	Pol bei }p=0\text{ daher zuerst invertieren}\nonumber\\
	\underline{Z}(p)&=\frac{3+4p^2+p^4}{16p+12p^3+2p^5}\nonumber\\
	\underline{Z}(p)&=\left(3+4p^2+p^4\right):\left(16p+12p^3+2p^5\right)=\frac{3}{16p}+\frac{\frac{7}{4}p^2+\frac{5}{8}p^4}{16p+12p^3+2p5}\nonumber\\
	\underline{Y}_1(p)&=\left(16p+12p^3+2p^5\right):\left(\frac{7}{4}p^2+\frac{5}{8}p^4\right)=\frac{64}{7p}+\frac{\frac{44}{7}p^3+2p^5}{\frac{7}{4}p^2+\frac{5}{8}p^4}\nonumber\\
	\underline{Z}_2(p)&=\left(\frac{7}{4}p^2+\frac{5}{8}p^4\right):\left(\frac{44}{7}p^3+2p^5\right)=\frac{49}{176p}+\frac{\frac{3}{44}p^4}{\frac{44}{7}p^3+2p^5}\nonumber\\
	\underline{Y}_3(p)&=\left(\frac{44}{7}p^3+2p^5\right):\left(\frac{3}{44}p^4\right)=\frac{44^2}{21p}+\frac{2p^5}{\frac{3}{44}p^4}\nonumber\\
	\underline{Z}_4(p)&=\frac{3}{44}p^4:2p^5=\frac{3}{88p}\nonumber\\
	\underline{Z}(p)&=\frac{3}{16p}+\frac{1}{\frac{64}{7p}+\frac{1}{\frac{49}{176p}+\frac{1}{\frac{44^2}{21p}+\frac{1}{\frac{3}{88p}}}}}\nonumber
\end{align}
%TODO: He is the motherfucking pterodactyl!
\tikzexternaldisable
\begin{circuitikz}[scale=4, european, american inductors, yscale=0.8]
\ctikzset{bipoles/length=1.2cm}
\draw (0,0)
	to[C=$\frac{16}{3}F$, *-] (1,0)
	to[C=$\frac{176}{49}F$] (2,0)
	to[C=$\frac{88}{3}F$] (3,0)
	to[short] (3,-1)
	to[short, -*] (0,-1)
	(1,0) to [L=$\frac{7}{64}H$] (1,-1)
	(2,0) to [L=$\frac{21}{1936}$] (2,-1)
	;
\draw[dashed, very thick, color=green, ->]
	(0.2,-1.2) |- +(0.2,0.7)
	;
\draw[dashed, very thick, color=green, ->]
	(0.7,-1.2) |- +(0.2,0.7)
	;
\draw[dashed, very thick, color=green, ->]
	(1.3,-1.2) |- +(0.2,0.7)
	;
\draw[dashed, very thick, color=green, ->]
	(1.7,-1.2) |- +(0.2,0.7)
	;
\draw[dashed, very thick, color=green, ->]
	(2.3,-1.2) |- +(0.2,0.7)
	;
\draw
	(0.2,-1.3) node[draw=none, fill=none, color=green] {$\underline{Z}(p)$}
	(0.7,-1.3) node[draw=none, fill=none, color=green] {$\underline{Y}_1(p)$}
	(1.3,-1.3) node[draw=none, fill=none, color=green] {$\underline{Z}_3(p)$}
	(1.7,-1.3) node[draw=none, fill=none, color=green] {$\underline{Y}_3(p)$}
	(2.3,-1.3) node[draw=none, fill=none, color=green] {$\underline{Z}_4(p)$}
;
\end{circuitikz}
\tikzexternalenable
\\
Alle 4 Zerlegungen ergeben äquivalente ($\neq$ duale!) Schaltungen!\\
Man kann auch während der Zerlegung die Art wechseln.\\