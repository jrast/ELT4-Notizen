\section{Reaktanz Eintore}
Ein Reaktanz Eintor ist eine Schaltung aus lauter $L$ und $C$ mit genau zwei
Anschlüssen.

Skript von Herrn Neubauer, Seiten 27ff

Bei $\omega = 0$ und $\omega \rightarrow \infty$ hat $X(\omega)$ entweder einen
Pol oder eine Nullstelle.

\textbf{Es gibt folgende 4 RET-Typen:}

\begin{tabular}{llll}
Typ & $\omega=0$ & $\omega \rightarrow \infty$&\\
L & 0 & X & L-Symbol\\
C & X & 0 & C-Symbol\\
S & X & X & LC-Seriell\\
P & 0 & 0 & LC-Parallel\\
\end{tabular}

0 = Nullstelle\\
X = Polstelle\\

$$\underline{Z}(p)=\frac{a_np^n+a_{n-2}p^{n-2}+\ldots}{b_mp^m+b_{m-2}p^{m-2}}$$\\
$p\rightarrow \infty$ höchsten Potenzen entscheidend\\
\hspace*{5pt}$n>m$ (Zählergrad $>$ Nennergrad): Pol\\
\hspace*{10pt}$\frac{a_n}{b_m}=L_{\infty}$\\
\hspace*{5pt}$n<m$ (Zählergrad $<$ Nennergrad): Nullstelle\\
\hspace*{10pt}$\frac{a_n}{b_m}=\frac{1}{C_{\infty}}$\\
$p\rightarrow 0$ tiefste Potenzen massgebend\\
\hspace*{5pt}n ungerade
$\frac{\ldots+a_3p^3+a_1p}{\ldots+b_2p^2+b_0}\rightarrow$ Nullstelle für $p\rightarrow 0$\\
\hspace*{10pt}$\frac{a_1}{b} = L_0$ ersetzung bei ganz tiefen Frequenzen.\\
\hspace*{5pt}n gerade $\frac{\ldots+a_2p^2+a_0}{\ldots+b_3p^3+b_1p}$\\
\hspace*{10pt}$\frac{a_0}{b_1}=\frac{1}{C_0}$ ersetzung bei ganz tiefen
Frequenzen.\\


\subsection{Bestimmung des RET-Typs anhand der Schaltung}
\begin{tabular}{llll}
L-Symbol & L-Typ & L-Klemmenkreis & L-Klemmen Trennbündel\\
& & Nullstelle bei $\omega = 0$ & Pol bei $\omega \rightarrow \infty$ \\
C-Symbol & C-Typ & C-Klemmenkreis & C-Klemmen Trennbündel\\
& & Nullstelle bei $\omega \rightarrow \infty$ & Pol bei $\omega \rightarrow
0$\\
LC-Symbol & S-Typ & C-Klemmentrennbündel & L-Klemmen Trennbündel\\
LC-Parallel & P-Typ & C-Klemmenkreis & L-Klemmenkreis\\
\end{tabular}

\subsection{RET-Synthese}
%schöne kraafick einfügen. Baum malen, was ganz tolles neues
\textbf{geg:} RET-Funktion\\
\textbf{ges:} zugehörige Schaltung\\
\subsubsection{Synthese mittels Partialbruchzerlegung}
\textbf{a) Zerlegung nach Polen der Impedanz}\\
Beispiel: $\underline{Z}(p) = \frac{p^4+4p^2+3}{2p^5+12p^3+16p}$\\
\begin{tabular}{lll}
	Vorabklärung & $p=0$ & Pol \\
	& $p \rightarrow \infty$ & Nullstelle \\
	& 5 Elemente & wegen $p^5$ \\
\end{tabular}
%Fussmalergrafik
In der Grafik sieht man dass es sich bei $\omega \rightarrow 0$ und $\omega
\rightarrow \infty$ wie ein C verhält. Draus folgt dass es ein C-Typ ist.\\
% Schaltung: - C - LCPar - LCPar -
$\underline{Z}_K=\frac{1}{pC_k+\frac{1}{pL_k}} = \frac{pL_k}{p^2L_KC_K+1}$\\
$\underline{Z}(p)=\frac{p^4+4p^2+3}{2p^5+12p^3+16p}$\\
Nennergrad $>$ Zählergrad: OK\\
%Folgende Zeile in align umbauen?
$\underline{N}(p)=2p^5+12p^3+16p=2p(p^4+6p^2+8)=2p(p^2+4)(p^2+2)$\\
Nullstellen vom Nenner bei: $p=0,\ p^2=-4,\ p^2=-2$\\
$\underline{Z}_K=\frac{A}{p}+\frac{Bp}{p^2+4}+\frac{Cp}{p^2+2}$\\
$\rightarrow A,B,C$ bestimmen, Gleichung mit ganzem Nenner multiplizieren\\
$p^4+4p^2+3=2A(p^2+4)(p^2+2)+Bp\cdot 2p(p^2+2)+Cp(2p)(p^2+4)$\\
$p=0:\ 3=2A\cdot 4 \cdot 2 \Rightarrow A = \frac{3}{16}$\\
$p^2=-4:\ 16-16+3=B\cdot 2\cdot (-4)(-2) \Rightarrow B=\frac{3}{16}$\\
%undergeschwoffene klammern 3 und 16
$p^2=-2: \ldots \Rightarrow C=\frac{1}{8}$\\
$\underline{Z}(p)=\frac{\frac{3}{16}}{p}+\frac{\frac{3}{16}p}{p^2+4}+\frac{\frac{1}{8}p}{p^2+2}$\\
$=\frac{1}{\frac{16}{3}p}+\frac{1}{\frac{16}{3}p+\frac{4\cdot\frac{16}{3}}{p}}+\frac{1}{8p+\frac{2\cdot8}{p}}$\\
$=\frac{1}{\frac{16}{3}p}+\frac{1}{\frac{16}{3}p+\frac{1}{\frac{3}{64}p}}+\frac{1}{8p+\frac{1}{\frac{1}{16}p}}$\\
%bildli c - clpar - clpar
%c = 16/3F
%cpar1 = 16/3f
%lpar1 = 3/64h
%cpar2 = 8f
%lpar2 = 1/16h
\textbf{b) Zerlegung nach Polen der Admittanz}\\
%Bildli parallel: c, lc, lc
Pole der Admittanz $\rightarrow \infty$ grosse Leitfähigkeit (bei
Resonanzfrequenz)\\
$\underline{Y}(p)=\frac{1}{\underline{Z}(p)}=\frac{2p^5+12p^3+16p}{p^4+4p^2+3}$\\
Zählergrad $>$ Nennergrad: zuerst ausdividieren, Rest ist eine echt gebrochen
rationale Funktion.\\
$\underline{Y}(p)=(2p^5+12p^3+16p):(p^4+4p^2+3)=2p+\frac{4p^3+10p}{p^4+4p^2+3}$\\
$\underline{Y}_{rest}(p)=\frac{4p^3+10p}{p^4+4p^2+3}=\frac{4p^3+10p}{(p^2+3)(p^2+1)}=\frac{Ap}{p^2+3}+\frac{Bp}{p^2+1}$\\
$4p^3+10p=Ap(p^2+1)+Bp(p^2+3)$\\
$4p^2+10=A(p^2+1)+B(p^2+3)$\\
$p^2=-3:\ -12+10=A(-2) \Rightarrow A=1$\\
%untergeschwoffen -2
$p^2=-1:\ -4+10=B(2) \Rightarrow B=3$\\
%untergeschwoffen 6
$\underline{Y}(p)=2p+\frac{p}{p^2+3}+\frac{3p}{p^2+1}=2p+\frac{1}{p+\frac{1}{\frac{1}{3}p}}+\frac{1}{\frac{1}{3}p+\frac{1}{3p}}$\\
%Bildli von oben
%c=2f
%cs1=1/3F
%ls1=1h
%cs2=3f
%ls2=1/3f
\subsubsection{Synthese mittels Kettenbruchzerlegung}
\textbf{a) 1. Art} Man geht aus von unecht gebrochen rationaler Funktion und
spaltet Pol bei $\infty$ ab. Dann invertiert man, etc.\\
$\underline{Y}(p)=(2p^5+12p^3+16p):(p^4+4p^2+3)=2p+\frac{4p^3+10p}{p^4+4p^2+3}$\\
$\underline{Z}_1=(p^4+4p^2+3):(4p^3+10p)=\frac{1}{4}p+\frac{\frac{3}{2}p^2+3}{4p^3+10p}$\\
$\underline{Y}_2=(4p^3+10p):(\frac{3}{2}p^2+3)=\frac{8}{3}p+\frac{2p}{\frac{3}{2}p^2+3}$\\
$\underline{Z}_3=(\frac{3}{2}p^2+3):(2p)=\frac{3}{4}p+\frac{3}{2p}=\frac{3}{4}p+\frac{1}{\frac{2}{3}p}$\\
$\frac{3}{4}=L$\\
$\frac{2}{3}=C$\\
%Bildli: parallel: c - lc - lc
%c = 2F
%
