\section{Zweitore}
Skript\\
\subsection{Umrechnung von Zweitorparametern}
Bild1\\
\textbf{Bsp/Aufgabe:} Man berechne $[\underline{A}]$ aus $[\underline{Z}]$\\

\begin{align}
	\underline{U}_1=\underline{Z}_{11}\underline{I}_1+\underline{Z}_{12}\underline{I}_2
	&
	\underline{U}_1=\underline{A}_{11}\underline{U}_2+\underline{A}_{12}(\underline{-I}_2)\nonumber\\
	\underline{U}_2=\underline{Z}_{21}\underline{I}_1+\underline{Z}_{22}\underline{I}_2
	&
	\underline{U}_1=\underline{A}_{21}\underline{U}_2+\underline{A}_{22}(\underline{-I}_2)\nonumber\\
	\underline{I}_1=\frac{1}{\underline{Z}_{21}}\left(\underline{U}_2-\underline{Z}_{22}\underline{I}_2\right)=\frac{1}{\underline{Z{21}}}\underline{U}_2+\frac{\underline{Z}_{22}}{\underline{Z}_{11}}\left(-\underline{I}_2\right)\nonumber\\
	\underline{U}_1=\frac{\underline{Z}_{11}}{\underline{Z}_{21}}\underline{U}_2+\underline{Z}_{11}\frac{\underline{Z}_{12}}{\underline{Z}_{21}}\left(-\underline{I}_2\right)-\underline{Z}_{12}\left(-\underline{I}_2\right)\nonumber\\
	\underline{U}_1=\frac{\underline{Z}_{11}}{\underline{Z}_{21}}\underline{U}_2+\left(\underline{Z}_{11}\frac{\underline{Z}_{22}}{\underline{Z}_{21}}-\underline{Z}_{12}\right)\cdot\left(-\underline{I}_2\right)\nonumber\\
	%Underbrace unter die zweitletzte klammer in letzter zeile
	\frac{\underline{Z}_{11}\underline{Z}_{22}-\underline{Z}_{12}\underline{Z}_{21}}{\underline{Z}_{21}}=\frac{det[Z]}{\underline{Z}_{21}}\nonumber\\
	[\underline{A}]=\frac{1}{\underline{Z}_{21}}
	\begin{bmatrix}
		\underline{Z}_{11} & det[\underline{Z}]\\
		1 & \underline{Z}_{22}
	\end{bmatrix}\nonumber
\end{align}
\subsection{Parameter-Bestimmung}
\textbf{Bsp} Bild 2\\
\textbf{Ges:} $\underline{Z}$-Matrix\\
%TODO Z-Matrix formel
\begin{align}
	\underline{U}_1=\underline{Z}_{11}\underline{I}_1+\underline{Z}_{12}\underline{I}_2\nonumber\\
	\underline{U}_2=\underline{Z}_{21}\underline{I}_1+\underline{Z}_{22}\underline{I}_2\nonumber
	\end{align}\\
Zuerst allgemein als T-Glied\\
Bild 3\\
$\underline{Z}_{11}=?$
Falls $I_2=0$ (Ausg. offen)\\
\begin{align}
	%Vor Rightarrow einen strich runter, unden ma strich \underline{I}_2=0
	\underline{U}_1=\underline{Z}_11\underline{I}_1 \Rightarrow
	\underline{Z}_{11}=\frac{\underline{U}_1}{\underline{I}_1}=\underline{Z}_1+\underline{Z}_2\nonumber\\
	=\text{(primäre) Leerlaufimpedanz }\underline{Z}_{1l}\nonumber\\
	\underline{Z}_{12}=?\nonumber\\
	\underline{U}_1=\underline{Z}_{11}\underline{I}_1+\underline{Z}_{12}\underline{I}_2\nonumber\\
	%Nach frac strich runter \underline{I}_1=0
	\text{Falls} I_1=0 \text{Eingang offen}:
	\underline{Z}_{12}=\frac{\underline{U}_1}{\underline{I}_2s}\nonumber
\end{align}
Bild 4\\
\begin{align}
	\underline{U}_1=\underline{Z}_2\underline{I}_2 \Rightarrow
	\frac{\underline{U}_1}{\underline{I}_2}=\underline{Z}_3=\underline{Z}_{12}\nonumber\\
	\underline{Z}_{22}=?
	\underline{U}_2=\underline{Z}_{21}\underline{I}_1+\underline{Z}_{22}\underline{I}_2\nonumber\\
	%TODO Nach 1. frac strich nach unten I_1=0
	\underline{Z}_{22}=\frac{\underline{U}_2}{\underline{I}_2}=\underline{Z}_2+\underline{Z}_3\nonumber\\
	\underline{Z}_{21}=?\nonumber\\
	%TODO Nach 1. frac strich nach unten I_2=0
	\underline{Z}_{21}=\frac{\underline{U}_2}{\underline{I}_1}=\underline{Z}_3\nonumber\\
	[\underline{Z}]=
	\begin{bmatrix}
		\underline{Z}_1+\underline{Z}_3 & \underline{Z}_3\\
		\underline{Z}_3 & \underline{Z}_2+\underline{Z}_3
	\end{bmatrix}\nonumber\\
	[\underline{Z}]=
	\begin{bmatrix}
		R+j\omega L & R\\
		R & R+\frac{1}{j\omega C}
	\end{bmatrix}\nonumber
\end{align}
Titel?\\
Bild5\\
\begin{align}
	\underline{I}_1=\underline{Y}_11\underline{U}_1+\underline{Y}_12\underline{U}_2\nonumber\\
	\underline{I}_2=\underline{Y}_21\underline{U}_1+\underline{Y}_22\underline{U}_2\nonumber\\
	\underline{Y}_{11}=?\nonumber\\
	%strich nach frac U_2=0
	\underline{Y}_{11}=\frac{\underline{I}_1}{\underline{U}_1}\nonumber\\
	U_2=0: \text{ Kurzschluss am Ausgang}\nonumber\\
	\frac{\underline{I}_1}{\underline{U}_1}=\underline{Y}_1+\underline{Y}_3=\underline{Y}_{11}\nonumber\\
	%strich nach frac U_1=0
	\underline{Y}_{11}=\frac{\underline{I}_1}{\underline{U}_2}=-\underline{Y}_3\nonumber\\
\end{align}
Bild 6\\
Analog dazu:\\
\begin{align}
	\underline{Y}_{22}=\underline{Y}_2+\underline{Y}_3\nonumber\\
	\underline{Y}_{21}=\underline{Y}_3\nonumber\\
\end{align}
Daraus folgt:
\begin{align}
	[\underline{Y}]=
	\begin{bmatrix}
		\underline{Y}_1+\underline{Y}_3 & -\underline{Y}_3\\
		-\underline{Y}_3 & \underline{Y}_2+\underline{Y}_3
	\end{bmatrix}\nonumber
\end{align}
\textbf{Bsp}
Bild 7\\
\textbf{Ges:} $[\underline{A}=?]$\\
\begin{align}
	\underline{U}_1=\underline{A}_{11}\underline{U}_2+\underline{A}_{12}\left(-\underline{I}_2\right)\nonumber\\
	\underline{I}_1=\underline{A}_{21}\underline{U}_2+\underline{A}_{22}\left(-\underline{I}_2\right)\nonumber\\
	\underline{A}_{11}=?\nonumber\\
	%Nach frac strich runter \underline{I}_2=0 (Ausgang offen)
	\underline{A}_{11}=\frac{\underline{U}_1}{\underline{U}_2} \rightarrow\text{
	Quelle } \underline{U}_1 \text{anschliessen.}\nonumber\\
	%Vor = strich runter \underline{I}_2=0
	\frac{\underline{U}_2}{\underline{U}_1}=\frac{\frac{1}{j\omega
	C}}{R+\frac{1}{j\omega C}}=\frac{1}{1+j\omega RC}\nonumber\\
	\rightarrow \underline{A}_{11}=1+j\omega RC\nonumber\\
	\underline{A}_{12}=?\nonumber\\
	%nach = strich runter \underline{U}_2=0 (Kurzschluss beim Ausgang)
	\underline{A}_{12}=\frac{\underline{U}_1}{-\underline{{I}_2}}\nonumber
\end{align}
Bild 8\\
\begin{align}
	\underline{A}_{21}=?\nonumber\\
	%strich runter nach = \underline{I}_2=0
	\underline{A}_{21}=\frac{\underline{I}_1}{\underline{U}_2}\nonumber
\end{align}
Bild 9
\begin{align}
	\underline{U}_2=\frac{\underline{I}_1}{j\omega C}\nonumber\\
	\frac{\underline{I}_1}{\underline{U}_2}=j\omega C =
	\underline{A}_{21}\nonumber\\
	\underline{A}_{22}=?\nonumber\\
	%strich nach unten nach frac \underline{U}_2=0
	\underline{A}_{22}=\frac{\underline{I}_1}{-\underline{I}_2}=1\nonumber\\
	[A]=
	\begin{bmatrix}
		1+j\omega RC & R\\
		j\omega C & 1
	\end{bmatrix}\nonumber
\end{align}
Beispiel Seite 55\\
\subsection{Leerlauf und Kurzschlussimpedanzen}
\textbf{Def}Bild 10\\
$\underline{Z}_{1l}=$ Primäre Leerlaufimpedanz\\
$\underline{Z}_{1k}=$ Primäre Kurzschlussimpedanz\\
$\underline{Z}_{2l}=$ Sekundäre Leerlaufimpedanz\\
$\underline{Z}_{2k}=$ Sekundäre Kurzschlussimpedanz\\
\subsubsection{Zusammenhang mit Zweitorparametern}
\begin{align}
	\underline{Z}_{1l}:
	\underline{U}_1=\underline{Z}_11\underline{I}_1+\underline{Z}_{12}\underline{I}_2\nonumber\\
	\underline{Z}_{1l}=\frac{\underline{U_1}}{\underline{I}_1}=\underline{Z}_{11}\nonumber\\
	\underline{Z}_{1k}:
	\underline{I}_1=\underline{Y}_{11}\underline{U}_1+\underline{Y}_{12}\underline{U}_2\nonumber\\
	\underline{Z}_{1k}=\frac{\underline{U}_1}{\underline{I}_1}=\frac{1}{\underline{Y}_{11}},
	\underline{Y}_{1k}=\underline{Y}_{11}\nonumber
\end{align}
\subsection{Leerlauf und Kurzschluss-Übertragungsgrössen}
\textbf{Bsp} (Leerlauf-)Spannungsübersetzung vorwärts\\
Vorwärts bedeutet speisung am Eingang\\
Bild 11\\
\begin{align}
	%Nach frac strich runter \underline{I}_2=0
	\underline{V}_{U21}=\frac{\underline{U}_2}{\underline{U}_1}=?\nonumber\\
	\underline{U}_1=\underline{A}_{11}\underline{U}_{2}+\underline{A}_{12}\left(-\underline{I}_2\right)\nonumber\\
	%Nach 2. frac strich runter \underline{I}_2=0
	\frac{\underline{U}_2}{\underline{U}_1}=\frac{1}{\underline{A}_11}\nonumber\\
\end{align}
(Kurzschluss-)Übertragungsadmittanz vorwärts
$\frac{\underline{I}_2}{\underline{U}_1}$\\
Bild 12\\
\begin{align}
	
\end{align}