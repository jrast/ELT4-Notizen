\section{Zweitore}
\tikzsetexternalprefix{compiledFigures/zweitore/}
\index{Zweitore}
\index{Zweitore!Umrechnung}
Skript\\
\subsection{Umrechnung von Zweitorparametern}
\begin{tikzpicture}[thick, ->]
  \draw (0,0.2) -- (0.5,0.2);
  \draw (0,0.8) -- (0.5,0.8) node[above left] {$\underline{I}_1$};
  \draw (-0.1,0.7) -- (-0.1,0.3) node[left] {$\underline{U}_1$};
  \draw (2.5,0.8) -- (2,0.8) node[above right] {$\underline{I}_2$};
  \draw (2.5,0.2) -- (2,0.2);
  \draw (2.6,0.7) -- (2.6,0.3) node[right] {$\underline{U}_2$};
  \draw (0.5,0) rectangle	(2,1);
  \node at (1.25,0.5) {2Tor};
\end{tikzpicture}\\
\textbf{Bsp/Aufgabe:} Man berechne $[\underline{A}]$ aus $[\underline{Z}]$\\

\begin{align}
	\underline{U}_1&=\underline{Z}_{11}\underline{I}_1+\underline{Z}_{12}\underline{I}_2\nonumber\\
	\underline{U}_2&=\underline{Z}_{21}\underline{I}_1+\underline{Z}_{22}\underline{I}_2\nonumber\\
	\nonumber\\
	\underline{U}_1&=\underline{A}_{11}\underline{U}_2+\underline{A}_{12}(\underline{-I}_2)\nonumber\\
	\underline{U}_1&=\underline{A}_{21}\underline{U}_2+\underline{A}_{22}(\underline{-I}_2)\nonumber\\
	\underline{I}_1&=\frac{1}{\underline{Z}_{21}}\left(\underline{U}_2-\underline{Z}_{22}\underline{I}_2\right)=\frac{1}{\underline{Z{21}}}\underline{U}_2+\frac{\underline{Z}_{22}}{\underline{Z}_{11}}\left(-\underline{I}_2\right)\nonumber\\
	\underline{U}_1&=\frac{\underline{Z}_{11}}{\underline{Z}_{21}}\underline{U}_2+\underline{Z}_{11}\frac{\underline{Z}_{12}}{\underline{Z}_{21}}\left(-\underline{I}_2\right)-\underline{Z}_{12}\left(-\underline{I}_2\right)\nonumber\\
	\underline{U}_1&=\frac{\underline{Z}_{11}}{\underline{Z}_{21}}\underline{U}_2+\underbrace{\left(\underline{Z}_{11}\frac{\underline{Z}_{22}}{\underline{Z}_{21}}-\underline{Z}_{12}\right)}_{\frac{\underline{Z}_{11}\underline{Z}_{22}-\underline{Z}_{12}\underline{Z}_{21}}{\underline{Z}_{21}}=\frac{det[Z]}{\underline{Z}_{21}}}\cdot\left(-\underline{I}_2\right)\nonumber\\
	[\underline{A}]&=\frac{1}{\underline{Z}_{21}}
	\begin{bmatrix}
		\underline{Z}_{11} & det[\underline{Z}]\\
		1 & \underline{Z}_{22}
	\end{bmatrix}\nonumber
\end{align}
\subsection{Parameter-Bestimmung}
\textbf{Bsp} \tikzexternaldisable
\begin{circuitikz}[scale=2, european, american inductors]
\ctikzset{bipoles/length=1.2cm}
	\draw (0,1) to [L=L, i>^=$\underline{I}_1$] (1,1) to [C=C,
	i<=$\underline{I}_2$] (2,1);
	\draw (1,1) to [R=R] (1,0);
	\draw (0,0) -- (2,0);
	\draw (0,1) to[open, v>=$\underline{U}_1$] (0,0);
	\draw (2,1) to[open, v^>=$\underline{U}_2$] (2,0);
\end{circuitikz}
\tikzexternalenable\\
\textbf{Ges:} $\underline{Z}$-Matrix\\
%TODO Z-Matrix formel
\begin{align}
	\underline{U}_1=\underline{Z}_{11}\underline{I}_1+\underline{Z}_{12}\underline{I}_2\nonumber\\
	\underline{U}_2=\underline{Z}_{21}\underline{I}_1+\underline{Z}_{22}\underline{I}_2\nonumber
	\end{align}\\
Zuerst allgemein als T-Glied\\
\begin{circuitikz}[scale=2, european]
\ctikzset{bipoles/length=1.2cm}
	\draw (0,1) to [L=$Z_1$, i>^=$\underline{I}_1$] (1,1) to [L=$Z_2$,
	i<=$\underline{I}_2$] (2,1);
	\draw (1,1) to [L=$Z_3$] (1,0);
	\draw (0,0) -- (2,0);
	\draw (0,1) to[open, v>=$\underline{U}_1$] (0,0);
	\draw (2,1) to[open, v^>=$\underline{U}_2$] (2,0);
\end{circuitikz}\\
$\underline{Z}_{11}=?$
Falls $I_2=0$ (Ausg. offen)\\
\begin{align}
	\left. \underline{U}_1=\underline{Z}_{11}\underline{I}_1
	\right|_{\underline{I}_2=0}  \Rightarrow \underline{Z}_{11}=\frac{\underline{U}_1}{\underline{I}_1}=\underline{Z}_1+\underline{Z}_2\nonumber\\
	=\text{(primäre) Leerlaufimpedanz }\underline{Z}_{1l}\nonumber
\end{align}
$\underline{Z}_{12}=?$\\
\begin{align}
	\underline{U}_1&=\underline{Z}_{11}\underline{I}_1+\underline{Z}_{12}\underline{I}_2\nonumber\\
	\text{Falls } I_1&=0 \text{ Eingang offen}:\nonumber\\
	\underline{Z}_{12}&=\left.\frac{\underline{U}_1}{\underline{I}_2s}\right|_{\underline{I}_1=0}\nonumber
\end{align}
\tikzexternaldisable
\begin{circuitikz}[scale=2, european]
\ctikzset{bipoles/length=1.2cm}
	\draw (0,1) to [L=$Z_1$, i>^=$\underline{I}_1$] (1,1) to [L=$Z_2$,
	i<=$\underline{I}_2$] (2,1);
	\draw (1,1) to [L=$Z_3$] (1,0);
	\draw (0,0) -- (2,0);
	
	\draw (0,1) to[open, v>=$\underline{U}_1$] (0,0);
	\draw (2,1) to[sinusoidal current source=$\underline{I}_2$, i<_=$ $] (2,0);
\end{circuitikz}
\tikzexternalenable\\
\begin{align}
	\underline{U}_1=\underline{Z}_2\underline{I}_2 \Rightarrow
	\frac{\underline{U}_1}{\underline{I}_2}=\underline{Z}_3=\underline{Z}_{12}\nonumber
\end{align}
$\underline{Z}_{22}=?$\\
\begin{align}
	\underline{U}_2&=\underline{Z}_{21}\underline{I}_1+\underline{Z}_{22}\underline{I}_2\nonumber\\
	\underline{Z}_{22}&=\left.\frac{\underline{U}_2}{\underline{I}_2}\right|_{I_1=0}=\underline{Z}_2+\underline{Z}_3\nonumber
\end{align}
$\underline{Z}_{21}=?$\\
\begin{align}
	\underline{Z}_{21}&=\left.\frac{\underline{U}_2}{\underline{I}_1}\right|_{I_2=0}=\underline{Z}_3\nonumber\\
	[\underline{Z}]&=
	\begin{bmatrix}
		\underline{Z}_1+\underline{Z}_3 & \underline{Z}_3\\
		\underline{Z}_3 & \underline{Z}_2+\underline{Z}_3
	\end{bmatrix}\nonumber\\
	[\underline{Z}]&=
	\begin{bmatrix}
		R+j\omega L & R\\
		R & R+\frac{1}{j\omega C}
	\end{bmatrix}\nonumber
\end{align}
\tikzexternaldisable
\begin{circuitikz}[scale=2, european]
\ctikzset{bipoles/length=1.2cm}
	\draw (0,1) -- (0.5,1) to [R=$\underline{Y_3}$] (1.5,1) -- (2,1);
	\draw (0.5,1) to [R=$\underline{Y_1}$] (0.5,0);
	\draw (1.5,1) to [R=$\underline{Y_2}$] (1.5,0);
	\draw (0,0) -- (2,0);
\end{circuitikz}
\tikzexternalenable\\
\begin{align}
	\underline{I}_1=\underline{Y}_{11}\underline{U}_1+\underline{Y}_12\underline{U}_2\nonumber\\
	\underline{I}_2=\underline{Y}_{21}\underline{U}_1+\underline{Y}_22\underline{U}_2\nonumber
\end{align}
$\underline{Y}_{11}=?$\\
\begin{align}	
	\underline{Y}_{11}&=\left.\frac{\underline{I}_1}{\underline{U}_1}\right|_{U_2=0}\nonumber\\
	U_2&=0: \text{ Kurzschluss am Ausgang}\nonumber\\
	\frac{\underline{I}_1}{\underline{U}_1}&=\underline{Y}_1+\underline{Y}_3=\underline{Y}_{11}\nonumber\\
	\underline{Y}_{11}&=\left.\frac{\underline{I}_1}{\underline{U}_2}\right|_{U_1=0}=-\underline{Y}_3\nonumber
\end{align}
\tikzexternaldisable
\begin{circuitikz}[scale=2, european]
\ctikzset{bipoles/length=1.2cm}
	\draw (0,1) -- (0,0);
	\draw (0,1) to [short, i=$I_1$] (0.5,1) to [R=$\underline{Y_3}$] (1.5,1) --
	(2.5,1);
	\draw (0.5,1) to [R=$\underline{Y_1}$] (0.5,0);
	\draw (1.5,1) to [R=$\underline{Y_2}$] (1.5,0);
	\draw (2.5,1) to [sinusoidal voltage source = $\underline{U}_2$]	(2.5,0);
	\draw (0,0) -- (2.5,0);
\end{circuitikz}
\tikzexternalenable\\
Analog dazu:\\
\begin{align}
	\underline{Y}_{22}&=\underline{Y}_2+\underline{Y}_3\nonumber\\
	\underline{Y}_{21}&=\underline{Y}_3\nonumber
\end{align}
Daraus folgt:
\begin{align}
	[\underline{Y}]=
	\begin{bmatrix}
		\underline{Y}_1+\underline{Y}_3 & -\underline{Y}_3\\
		-\underline{Y}_3 & \underline{Y}_2+\underline{Y}_3
	\end{bmatrix}\nonumber
\end{align}
\textbf{Bsp}
\begin{circuitikz}[scale=2, european, american inductors]
\ctikzset{bipoles/length=1.2cm}
	\draw (0,1) to [R=R, i>^=$\underline{I}_1$] (1,1) to [short,
	i<=$\underline{I}_2$] (2,1);
	\draw (1,1) to [C=C] (1,0);
	\draw (0,0) -- (2,0);
	\draw (0,1) to[open, v>=$\underline{U}_1$] (0,0);
	\draw (2,1) to[open, v^>=$\underline{U}_2$] (2,0);
\end{circuitikz}\\
\textbf{Ges:} $[\underline{A}=?]$\\
\begin{align}
	\underline{U}_1=\underline{A}_{11}\underline{U}_2+\underline{A}_{12}\left(-\underline{I}_2\right)\nonumber\\
	\underline{I}_1=\underline{A}_{21}\underline{U}_2+\underline{A}_{22}\left(-\underline{I}_2\right)\nonumber
\end{align}
$\underline{A}_{11}=?$\\
\begin{align}
	\underline{A}_{11}&=\left.\frac{\underline{U}_1}{\underline{U}_2}\right|_{\underline{I}_2=0}
	\rightarrow\text{ Quelle } \underline{U}_1 \text{anschliessen.}\nonumber\\
	\left.\frac{\underline{U}_2}{\underline{U}_1}\right|_{\underline{I}_2=0}&=\frac{\frac{1}{j\omega
	C}}{R+\frac{1}{j\omega C}}=\frac{1}{1+j\omega RC}\nonumber\\
	\rightarrow \underline{A}_{11}&=1+j\omega RC\nonumber
\end{align}
$\underline{A}_{12}=?$\\
\begin{align}
	\left.\underline{A}_{12}=\right|_{\underline{U}_2=0}\frac{\underline{U}_1}{-\underline{{I}_2}}\nonumber
\end{align}
\tikzexternaldisable
\begin{circuitikz}[scale=2, european, american inductors]
\ctikzset{bipoles/length=1.2cm}
	\draw (0,1) to [R=R, i>^=$\underline{I}_1$] (1,1) to [short,
	i<=$\underline{I}_2$] (2,1);
	\draw (1,1) to [C=C] (1,0);
	\draw (0,0) -- (2,0);
	\draw (2,1) -- (2,0);
	\draw (0,1) to[open, v>=$\underline{U}_1$] (0,0);
\end{circuitikz}
\tikzexternalenable\\
$\underline{A}_{21}=?$\\
\begin{align}
	\underline{A}_{21}\left.=\right|_{\underline{I}_2=0}\frac{\underline{I}_1}{\underline{U}_2}\nonumber
\end{align}
\begin{circuitikz}[scale=2, european, american inductors]
\ctikzset{bipoles/length=1.2cm}
	\draw (0,1) to [R=R, i>^=$\underline{I}_1$] (1,1) to [short,
	i<=$\underline{I}_2$] (2,1);
	\draw (1,1) to [C=C] (1,0);
	\draw (0,0) -- (2,0);
	\draw (0,1) to[open, v>=$\underline{U}_1$] (0,0);
	\draw (2,1) to[open, v^>=$\underline{U}_2$] (2,0);
\end{circuitikz}\\
\begin{align}
	\underline{U}_2&=\frac{\underline{I}_1}{j\omega C}\nonumber\\
	\frac{\underline{I}_1}{\underline{U}_2}&=j\omega C =
	\underline{A}_{21}\nonumber
\end{align}
$\underline{A}_{22}=?$\\
\begin{align}
	\underline{A}_{22}&=\left.\frac{\underline{I}_1}{-\underline{I}_2}\right|_{\underline{U}_2=0}=1\nonumber\\
	[A]&=
	\begin{bmatrix}
		1+j\omega RC & R\\
		j\omega C & 1
	\end{bmatrix}\nonumber
\end{align}
Beispiel Seite 55\\
\subsection{Leerlauf und Kurzschlussimpedanzen}
\index{Zweitore!Leerlaufimpedanz}
\index{Zweitore!Kurzschlussimpedanz}
\textbf{Def}Bild 10\\
\begin{figure}[!h]
\centering
\subfloat[primäre Leerlaufimpedanz]{
	\begin{tikzpicture}[thick, ->]
  \draw (0,0.2) -- (0.5,0.2);
  \draw (0,0.8) -- (0.5,0.8);
  \draw (2.5,0.8) -- (2,0.8);
  \draw (2.5,0.2) -- (2,0.2);
  \draw (0.5,0) rectangle	(2,1);
  \node at (0.75,0.5) {1};
  \node at (1.75,0.5) {2};
  \node at (3,0.5) {offen};
  \draw (-0.5,0.5) -- (-0.1,0.5) node[left=8] {$Z_{1l}$};
\end{tikzpicture}
	\label{fig:zweitor:prim:leer}
}
\qquad
\subfloat[primäre Kurzschlussimpedanz]{
	\begin{tikzpicture}[thick]
  \draw [->] (0,0.2) -- (0.5,0.2);
  \draw [->] (0,0.8) -- (0.5,0.8);
  \draw [->] (2.5,0.8) -- (2,0.8);
  \draw [->] (2.5,0.2) -- (2,0.2);
  \draw (0.5,0) rectangle	(2,1);
  \node at (0.75,0.5) {1};
  \node at (1.75,0.5) {2};
  \draw (2.5,0.8) -- (2.5,0.2);
  \draw [->] (-0.5,0.5) -- (-0.1,0.5) node[left=8] {$Z_{1k}$};
\end{tikzpicture}
	\label{fig:zweitor:prim:kurz} 
}\\
\subfloat[sekundäre Leerlaufimpedanz]{
	\begin{tikzpicture}[thick, ->]
  \draw (0,0.2) -- (0.5,0.2);
  \draw (0,0.8) -- (0.5,0.8);
  \draw (2.5,0.8) -- (2,0.8);
  \draw (2.5,0.2) -- (2,0.2);
  \draw (0.5,0) rectangle	(2,1);
  \node at (0.75,0.5) {1};
  \node at (1.75,0.5) {2};
  \draw (3,0.5) -- (2.6,0.5) node[right=8] {$Z_{2l}$};
\end{tikzpicture}
	\label{fig:zweitor:sek:leer}
}
\qquad
\subfloat[sekundäre Kurzschlussimpedanz]{
	\begin{tikzpicture}[thick]
  \draw [->] (0,0.2) -- (0.5,0.2);
  \draw [->] (0,0.8) -- (0.5,0.8);
  \draw [->] (2.5,0.8) -- (2,0.8);
  \draw [->] (2.5,0.2) -- (2,0.2);
  \draw (0.5,0) rectangle	(2,1);
  \node at (0.75,0.5) {1};
  \node at (1.75,0.5) {2};
  \draw (0,0.2) -- (0,0.8);
  \draw [->] (3,0.5) -- (2.6,0.5) node[right=8] {$Z_{2l}$};
\end{tikzpicture}
	\label{fig:zweitor:sek:kurz} 
}
\caption{Definition Kurzschluss- und Leerlaufimpedanzen}
\label{fig:zweitor:impedanzen}
\end{figure}

\subsubsection{Zusammenhang mit Zweitorparametern}
\begin{align}
	\underline{Z}_{1l}:
	\underline{U}_1&=\underline{Z}_{11}\underline{I}_1+\underline{Z}_{12}\underline{I}_2\nonumber\\
	\underline{Z}_{1l}&=\frac{\underline{U_1}}{\underline{I}_1}=\underline{Z}_{11}\nonumber\\
	\underline{Z}_{1k}:
	\underline{I}_1&=\underline{Y}_{11}\underline{U}_1+\underline{Y}_{12}\underline{U}_2\nonumber\\
	\underline{Z}_{1k}&=\frac{\underline{U}_1}{\underline{I}_1}=\frac{1}{\underline{Y}_{11}},
	\underline{Y}_{1k}=\underline{Y}_{11}\nonumber
\end{align}
\subsection{Leerlauf und Kurzschluss-Übertragungsgrössen}
\index{Zweitore!Übertragungsgrössen}
\textbf{Bsp} (Leerlauf-)Spannungsübersetzung vorwärts\\
Vorwärts bedeutet speisung am Eingang\\
\begin{tikzpicture}[thick, ->]
  \draw (0,0.2) -- (0.5,0.2);
  \draw (0,0.8) -- (0.5,0.8);
  \draw (-0.1,0.7) -- (-0.1,0.3) node[left] {$\underline{U}_1$};
  \draw (2.5,0.8) -- (2,0.8) node[above right] {$\underline{I}_2=0$};
  \draw (2.5,0.2) -- (2,0.2);
  \draw (2.6,0.7) -- (2.6,0.3) node[right] {$\underline{U}_2$};
  \draw (0.5,0) rectangle	(2,1);
  \node at (1.25,0.5) {2Tor};
\end{tikzpicture}\\
\begin{align}
	\underline{V}_{U21}&=\left.\frac{\underline{U}_2}{\underline{U}_1}\right|_{\underline{I}_2=0}=?\nonumber\\
	\underline{U}_1&=\underline{A}_{11}\underline{U}_{2}+\underline{A}_{12}\left(-\underline{I}_2\right)\nonumber\\
	\frac{\underline{U}_2}{\underline{U}_1}&=\left.\frac{1}{\underline{A}_{11}}\right|_{\underline{I}_2=0}=\frac{\underline{Z}_{21}}{\underline{Z}_{11}}=\frac{-\underline{H}_{21}}{det[H]}\nonumber
\end{align}
(Kurzschluss-)Übertragungsadmittanz vorwärts
$\frac{\underline{I}_2}{\underline{U}_1}$\\
\begin{tikzpicture}[thick]
  \draw [->] (0,0.2) -- (0.5,0.2);
  \draw [->] (0,0.8) -- (0.5,0.8);
  \draw [->] (-0.1,0.7) -- (-0.1,0.3) node[left] {$\underline{U}_1$};
  \draw [->] (2.5,0.8) -- (2,0.8) node[above right] {$\underline{I}_2$};
  \draw [->] (2.5,0.2) -- (2,0.2);
  \draw (0.5,0) rectangle	(2,1);
  \draw (2.5,0.8) -- (2.5,0.2);
  \node at (1.25,0.5) {2Tor};
\end{tikzpicture}\\
\begin{align}
	\left.\frac{\underline{I}_2}{\underline{U}_1}\right|_{U_2=0}=?\nonumber\\
	\underline{I}_2=\underline{Y}_{21}\underline{U}_1\Rightarrow\frac{\underline{I}_2}{\underline{U}_1}=\underline{Y}_{21}\nonumber
\end{align}
Spannungsübersetzung rückwärts $\underline{V}_{U12}$\\
\begin{tikzpicture}[thick]
  \draw (0,0.2) -- (0.5,0.2);
  \draw (0,0.8) -- (0.5,0.8);
  \draw [->] (-0.1,0.7) -- (-0.1,0.3) node[left] {$\underline{U}_1$};
  \draw (2.5,0.8) -- (2,0.8);
  \draw (2.5,0.2) -- (2,0.2);
  \draw (2.5,0.8) -- (2.5,0.2);
  \draw (0.5,0) rectangle	(2,1);
  \draw (2.5,0.5) circle (0.2cm);
  \node at (3,0.5) {$\underline{U}_2$};
  \node at (1.25,0.5) {2Tor};
\end{tikzpicture}\\
\begin{align}
	\underline{V}_{U12}=\left.\frac{\underline{U}_1}{\underline{U}_2}\right|_{I_1=0}=\underline{H}_{12}=\frac{\underline{Z}_{12}}{\underline{Z}_{22}}\nonumber
\end{align}
Tabellen übernehmen? Umrechnung, herleitung etc\\
\subsection{Betrieb-Eingangs- und Übertragungsgrössen}
Last-(bzw Quellen-) Impedanz nicht $0$ oder $\infty$ \\
\textbf{Bsp} \begin{tikzpicture}[thick]
  \draw (0,0.2) -- (0.5,0.2);
  \draw (0,0.8) -- (0.5,0.8);
  \node at (-0.2,0.5) {$\underline{Z}_1$};
  \draw [->] (3.2,0.8) -- (2,0.8) node[above right] {$\underline{I}_2$};
  \draw (3.2,0.2) -- (2,0.2);
  \draw [->] (2.2,0.7) -- (2.2,0.3);
  \node at (2.5,0.5) {$\underline{U}_2$};
  \draw (3.2,0.8) -- (3.2,0.7);
  \draw (3.2,0.3) -- (3.2,0.2);
  \draw (3,0.7) rectangle (3.4,0.3);
  \node at (3.7,0.5) {$\underline{Z}_b$};
  \draw (0.5,0) rectangle	(2,1);
  \node at (1.25,0.5) {2Tor};
\end{tikzpicture}
\\
\subsubsection{Eingangsimpedanz}
\begin{align}
	\underline{Z}_1&=\frac{\underline{U}_1}{\underline{I}_1}=\frac{\underline{A}_{11}\underline{U}_{2}+\underline{A}_{12}\left(-\underline{I}_2\right)}{\underline{A}_{21}\underline{U}_2+\underline{A}_{22}\left(-\underline{I}_2\right)}\nonumber\\
	\underline{Z}_1&=\frac{\underline{A}_{11}+\underline{A}_{12}\frac{1}{\underline{Z}_b}}{\underline{A}_{21}+\underline{A}_{22}\frac{1}{\underline{Z}_b}}=\frac{\underline{A}_{11}\underline{Z}_b+\underline{A}_{12}}{\underline{A}_{21}\underline{Z}_b+\underline{A}_{22}}\nonumber
\end{align}
Ausgedrückt durch $\underline{Z}$-Parameter
\begin{align}
	\underline{Z}_1=\frac{\underline{Z}_{11}\underline{Z}_b+det[Z]}{\underline{Z}_b+\underline{Z}_{22}}\nonumber
\end{align}
\subsubsection{Betriebs-Spannungsübersetzung}
$\underline{V}_{Uba}=\frac{\underline{U}_{2}}{\underline{U}_{1}}$\\
\begin{tikzpicture}[thick]
  \draw (-0.5,0.2) -- (0.5,0.2);
  \draw (-0.5,0.8) -- (0.5,0.8);
  \draw [->] (0.3,0.7) -- (0.3,0.3);
  \node at (0,0.5) {$\underline{U}_1$};
  \draw (-0.5,0.8) -- (-0.5,0.2);
  \draw (-0.5,0.5) circle (0.2cm);
  \draw [->] (3.2,0.8) -- (2,0.8)  node[above right] {$\underline{I}_2$};
  \draw (3.2,0.2) -- (2,0.2);
  \draw [->] (2.2,0.7) -- (2.2,0.3);
  \node at (2.5,0.5) {$\underline{U}_2$};
  \draw (3.2,0.8) -- (3.2,0.7);
  \draw (3.2,0.3) -- (3.2,0.2);
  \draw (3,0.7) rectangle (3.4,0.3);
  \node at (3.7,0.5) {$\underline{Z}_b$};
  \draw (0.5,0) rectangle	(2,1);
  \node at (1.25,0.5) {2Tor};
\end{tikzpicture}\\
rechts vom bild:
$\underline{I}_{2}=\frac{-\underline{U}_{2}}{\underline{Z}_{b}}$\\
\begin{align}
	\underline{U}_{1}&=\underline{U}_{2}\left(\underline{A}_{11}+\frac{\underline{A}_{12}}{\underline{Z}_{b}}\right)\nonumber\\
	\frac{\underline{U}_{2}}{\underline{U}_{1}}&=\frac{1}{\underline{A}_{11}+\frac{\underline{A}_{12}}{\underline{Z}_{b}}}=\frac{\underline{Z}_{b}}{\underline{A}_{11}\underline{Z}_{b}+\underline{A}_{12}}=\underline{V}_{Uba}\nonumber
\end{align}
\begin{tikzpicture}[thick]
\tikzpicturedependsonfile{\currfilepath}
  \draw (-1,0.2) -- (0.5,0.2);
  \draw (-0.3,0.8) -- (0.5,0.8);
  \draw (-0.8,0.8) -- (-1,0.8);
  \draw [->] (0.1,0.7) -- (0.1,0.3);
  \node at (-0.2,0.5) {$\underline{U}_1$};
  \draw [->, dashed] (0.3,0.5) -- (0.8,0.5);
  \draw [dashed] (0.3,0.5) -- (0.3,0) node [below] {$\underline{Z}_1$};
  \draw (-0.8,0.7) rectangle (-0.3,0.9) node [above left] {$\underline{Z}_a$};
  \draw (-1,0.8) -- (-1,0.2);
  \draw (-1,0.5) circle (0.2cm);
  \draw [->] (3.2,0.8) -- (2,0.8)  node[above right] {$\underline{I}_2$};
  \draw (3.2,0.2) -- (2,0.2);
  \draw [->] (2.2,0.7) -- (2.2,0.3);
  \node at (2.5,0.5) {$\underline{U}_2$};
  \draw (3.2,0.8) -- (3.2,0.7);
  \draw (3.2,0.3) -- (3.2,0.2);
  \draw (3,0.7) rectangle (3.4,0.3);
  \node at (3.7,0.5) {$\underline{Z}_b$};
  \draw (0.5,0) rectangle	(2,1);
  \node at (1.25,0.5) {2Tor};
\end{tikzpicture}\\
\begin{align}
 	\frac{\underline{U}_{2}}{\underline{U}_{1}}&=?\nonumber\\
	\underline{U}_{1}&=\underline{U}_{q}\frac{\underline{Z}_{1}}{\underline{Z}_{a}+\underline{Z}_{1}}\nonumber\\
	\text{daraus: }
	\frac{\underline{U}_{2}}{\underline{U}_{q}}&=\frac{\underline{U}_{2}}{\underline{U}_{1}}\frac{\underline{U}_{1}}{\underline{U}_{q}}\nonumber
\end{align}
Skript Seite 64 (Kaskaden- oder Kettenschaltung)\\
Skript Seite 65 (Parallelschaltung)\\
Parallelschaltung Achtung: Beim Zusammenschalten dürfen keine Zweige
kurzgeschlossen oder stromlos werden\\
\subsection{Zusammenschaltregel von Brun}
$\underline{Z} = \underline{Z}^\prime + \underline{Z}^{\prime\prime}$\\
\begin{figure}[!h]
\centering
\subfloat[Falsche Parallelschaltung]{
	\begin{tikzpicture}[thick]
  %Unteres Viereck
  \draw (0.5,0) rectangle (4.5,3);
    \draw (0.5,0.5) -- (4.5,0.5);
    \draw (0.5,2.5) -- (1,2.5);
    \draw (2,2.5) -- (3,2.5);
    \draw (4,2.5) -- (4.5,2.5);
    \draw (1,2.8) rectangle (2,2.2);
    \draw (3,2.8) rectangle (4,2.2);
    \draw (2.5,2.5) -- (2.5,2);
    \draw (2.2,2) rectangle (2.8,1);
    \draw (2.5,0.5) -- (2.5,1);
  %Oberes Viereck
  \draw (0.5,3.5) rectangle (4.5,6.5);
      \draw (0.5,4) -- (4.5,4);
      \draw (0.5,6) -- (1,6);
      \draw (2,6) -- (3,6);
      \draw (4,6) -- (4.5,6);
      \draw (1,6.3) rectangle (2,5.7);
      \draw (3,6.3) rectangle (4,5.7);
      \draw (2.5,6) -- (2.5,5.5);
      \draw (2.2,5.5) rectangle (2.8,4.5);
      \draw (2.5,4) -- (2.5,4.5);
  %Verbindungen
    \draw [red] (0,0.5) -- (0.5,0.5);
    \draw [red] (0.2,2.5) -- (0.5,2.5);
    \draw [red] (0.2,2.5) -- (0.2,4);
    \draw [red] (0.2,4) -- (0.5,4);
    \draw [red] (0,6) -- (0.5,6);
    %Rechts
    \draw [red] (5,0.5) -- (4.5,0.5);
    \draw [red] (4.8,2.5) -- (4.5,2.5);
    \draw [red] (4.8,2.5) -- (4.8,4);
    \draw [red] (4.8,4) -- (4.5,4);
    \draw [red] (5,6) -- (4.5,6);
\end{tikzpicture}
	\label{fig:zweitor:regelvbrun:falsch}
}
\qquad
\subfloat[Richtige Parallelschaltung]{
	\tikzsetnextfilename{RegelvBrunRichtig}
\begin{tikzpicture}[thick]
  %Unteres Viereck
  \draw (0.5,0) rectangle (4.5,3);
    \draw (0.5,2.5) -- (4.5,2.5);
    \draw (0.5,0.5) -- (1,0.5);
    \draw (2,0.5) -- (3,0.5);
    \draw (4,0.5) -- (4.5,0.5);
    \draw (1,0.8) rectangle (2,0.2);
    \draw (3,0.8) rectangle (4,0.2);
    \draw (2.5,2.5) -- (2.5,2);
    \draw (2.2,2) rectangle (2.8,1);
    \draw (2.5,0.5) -- (2.5,1);
  %Oberes Viereck
  \draw (0.5,3.5) rectangle (4.5,6.5);
      \draw (0.5,4) -- (4.5,4);
      \draw (0.5,6) -- (1,6);
      \draw (2,6) -- (3,6);
      \draw (4,6) -- (4.5,6);
      \draw (1,6.3) rectangle (2,5.7);
      \draw (3,6.3) rectangle (4,5.7);
      \draw (2.5,6) -- (2.5,5.5);
      \draw (2.2,5.5) rectangle (2.8,4.5);
      \draw (2.5,4) -- (2.5,4.5);
  %Verbindungen
    \draw [red] (0,0.5) -- (0.5,0.5);
    \draw [red] (0.2,2.5) -- (0.5,2.5);
    \draw [red] (0.2,2.5) -- (0.2,4);
    \draw [red] (0.2,4) -- (0.5,4);
    \draw [red] (0,6) -- (0.5,6);
    %Rechts
    \draw [red] (5,0.5) -- (4.5,0.5);
    \draw [red] (4.8,2.5) -- (4.5,2.5);
    \draw [red] (4.8,2.5) -- (4.8,4);
    \draw [red] (4.8,4) -- (4.5,4);
    \draw [red] (5,6) -- (4.5,6);
\end{tikzpicture}
	\label{fig:zweitor:regelvbrun:richtig}
}\\
\caption{Zusammenschaltregel von Brun}
\label{fig:zweitor:regelvbrun}
\end{figure}

\subsection{Reziprozität}
$\Rightarrow$ Ausgangsgrössen sind unabhängig von der Betriebsrichtung, sofern
$\underline{Z}_{Q}=\underline{Z}_{L} \left(=\underline{Z}_{0}\right)$\\
Versuchsanordnung:\\
Bild 7\\
Für das erweiterte Zweitor gilt:\\
\begin{align}
	\begin{bmatrix}
		U_1'\\
		U_2'
	\end{bmatrix}
	=	
	\begin{bmatrix}
		\underline{Z}_{11}+\underline{Z}_{0} & \underline{Z}_{12}\\
		\underline{Z}_{21} & \underline{Z}_{22}+\underline{Z}_0
	\end{bmatrix}
	\begin{bmatrix}
		\underline{I}_1'\\
		\underline{I}_2'
	\end{bmatrix}\nonumber
\end{align}
Übertragungsadmittanz vorwärts $\frac{\underline{I}_{2}'}{\underline{U}_{1}'}=\underline{Y}_{21}'
	(\underline{U}_{2}'=0)$\\
Übertragungsadmittanz rückwärts
$\frac{\underline{I}_{1r}'}{\underline{U}_{2r}'}=\underline{Y}_{12}'
(\underline{U}_{1}'=0)$\\
Betriebsbedingungen $\underline{U}_{1}'=\underline{U}_{2r}'=\underline{U}_{q}$\\
Reziprozitätsbedingungen $\underline{I}_{2}'=\underline{I}_{1r}'$\\
Daraus folgt: $\underline{Y}_{21}'=\underline{Y}_{12}'$\\
Falls Tabelle 1 digitalisiert wird hier korrekt verweisen!\\
Aus Tabelle 1 folgt:\\
\begin{align}
	\frac{\underline{Z}_{21}'}{det[\underline{Z}']}=\frac{-\underline{Z}_{12}'}{det[\underline{Z}']}\nonumber\\
	\Rightarrow \underline{Z}_{21}'=\underline{Z}_{12}'\nonumber\\
	\text{Reziprozitätsbedingungen:}\nonumber\\
	\boxed{\underline{Z}_{21}=\underline{Z}_{12}}\nonumber\\
	\text{bzw }\boxed{\underline{Y}_{21}=\underline{Y}_{12}}\nonumber\\
	\boxed{det[\underline{A}]=1}\nonumber
\end{align}
Beispiel Bild 8\\
Skript Seite 57\\

