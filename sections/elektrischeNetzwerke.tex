\section{Elektrische Netzwerke, Systeme}
\index{Elektrische Netzwerke|see{Elektrische Systeme}}
\index{Elektrische Systeme}
Skript Herr Neubauer Seite 36-37.
Netzwerk-Klassifizierung


\subsection{Netzwerk-Klassifizierung}
\paragraph{Auflösung}
%TODO überarbeiten
\begin{tabular}{lll}
& kontinuierlich & diskret \\
Elemente & räumlich verteilt & konzentriert \\
Signal-Änderung & zeitlich stetig & intervall-bedingt\\
& wertmässig analog & digital\\
\end{tabular}
\paragraph{Netzwerk-Elemente}
R-, RC-, RL-, RLC-Netzwerke
\paragraph{Ein- und Ausgänge}
\begin{tabular}{llll}
Bezeichnung & Eintor=Zweipol & Zweitor=Vierpol & N-tor=2N-pol \\
Zweigpaare & 1 & 2 & N\\
\end{tabular}\\
\ldots führen aus dem elektrischen Netzwerk heraus.
\paragraph{Verwendung}
Filter, Regelsysteme, Signalformer, Nachrichtensysteme, Niederspannungsnetze
u.a.
\paragraph{Systemeigenschaften}
%TODO Abschreiben??????

\subsection{Dualität}
\subsubsection{Duale Zweipole}
$\rightarrow$ vertauschte Rollen von Spannung und Strom\\
%TODO Tabelle überarbeiten. 4 Spalten 1. Symbol, leftrightarrow, 2. Symbol,
% erklärung
\begin{tabular}{ll}
$u=L\frac{di}{dt} \leftrightarrow i=C\frac{du}{dt}$
& L und C sind zueinander dual\\
$u=R\cdot i \leftrightarrow i=G\cdot u$ &
R und G sind zueinander dual\\
%TODO Symbole!
$U_q Spannungsquelle \leftrightarrow i_q Stromquellensymbol$
& Strom- und Spannungsquellen sind dual\\
\end{tabular}
\subsubsection{Duale Schaltungen}
Werden durch Gleichungen mit derselben "`mathematischen Struktur"' beschrieben.
Alle Ströme im einen Netzwerk sind proportional zur entsprechenden Spannung im
andern Netzwerk und umgekehrt.\\
Aus $\Sigma I=0$ wird $\Sigma U=0$ und umgekehrt\\
Aus Serieschaltungen werden Parallelschaltungen und umgekehrt\\
Aus Knoten werden Maschen und umgekehrt\\
\paragraph{Beispiel}
%TODO Bilder
Links:
\begin{align}
U_1+U_2=U_q\nonumber\\
U2=Uq\frac{R_2}{R_1+R_2}\nonumber\\
\end{align}

Rechts:
\begin{align}
I_1'+I_2'=I_q'\nonumber\\
I_2'=I_q'\frac{G_2'}{G_1'+G_2'}\nonumber\\
\end{align}

\subsubsection{Dualfaktor D}
Anpassung des Impedanzlevels, z.B.\\
alle Ströme $:1000$,\\
alle Spannungen $x1000$\\
$\rightarrow$ alle Widerstände (Imped.) $x10^6$\\
$\rightarrow$ alle Leitwerte (Admitt.) $:10^6$\\
Berechnung der dualen Grössen:\\
\begin{align}
	I_q'&=\frac{U_q}{1000\Ohm}\nonumber\\
	U'&=I\cdot 1000\Ohm\nonumber\\
	G'&=\frac{R}{(1000\Ohm)^2}\nonumber\\
	R'&=G(1000\Ohm)^2\nonumber
\end{align}
$1000\Ohm=$ Dualfaktor D, kann beliebig gewählt werden\\
\begin{tabular}{ll}
	$L \leftrightarrow C^2$ & $\underline{Y}'=\frac{\underline{Z}}{D^2}$\\
	& $j\omeg C'=\frac{j\omega L}{D^2} \Rightarrow C'=\frac{L}{D^2}$\\
	& $L'=C D^2$????????
\end{tabular}
Skript Seite 37
