\section{Elektrische Netzwerke, Systeme}
\index{Elektrische Netzwerke|see{Elektrische Systeme}}
\index{Elektrische Systeme}
Skript Herr Neubauer Seite 36-37.
Netzwerk-Klassifizierung


\subsection{Netzwerk-Klassifizierung}
\paragraph{Auflösung}
\begin{tabular}{|l|l|l|}
	\hline
	& kontinuierlich & diskret \\
	\hline
	Elemente & räumlich verteilt & konzentriert \\
	\hline
	Signal-Änderung & zeitlich stetig & intervall-bedingt\\
	& wertmässig analog & digital\\
	\hline
\end{tabular}
\paragraph{Netzwerk-Elemente}
R-, RC-, RL-, RLC-Netzwerke
\paragraph{Ein- und Ausgänge}
\begin{tabular}{llll}
Bezeichnung & Eintor=Zweipol & Zweitor=Vierpol & N-tor=2N-pol \\
Zweigpaare & 1 & 2 & N\\
\end{tabular}\\
\ldots führen aus dem elektrischen Netzwerk heraus.
\paragraph{Verwendung}
Filter, Regelsysteme, Signalformer, Nachrichtensysteme, Niederspannungsnetze
u.a.
\paragraph{Systemeigenschaften}
Aktiv – passiv, linear – nichtlinear, zeitvariant – zeitinvariant,
ohne Speicher – mit Speicher, minimalphasig – nicht minimalphasig\\


Die allgemeine Theorie der elektrischen Netzwerke begründet sich auf Arbeiten von Foster,
Cauer, Brune, Bode, Darlington, Piloty, Bader, Guillemin, Wagner, Heaviside und anderen.

\subsection{Dualität}
\subsubsection{Duale Zweipole}
$\rightarrow$ vertauschte Rollen von Spannung und Strom\\
\begin{tabular}{llll}
$u=L\frac{di}{dt}$&$ \leftrightarrow$&$ i=C\frac{du}{dt}$
& L und C sind zueinander dual\\
$u=R\cdot i $&$ \leftrightarrow $&$ i=G\cdot u$ &
R und G sind zueinander dual\\
$U_q $&$ \leftrightarrow $&$ i_q $
& Strom- und Spannungsquellen sind dual\\
\end{tabular}

\subsubsection{Duale Schaltungen}
Werden durch Gleichungen mit derselben "`mathematischen Struktur"' beschrieben.
Alle Ströme im einen Netzwerk sind proportional zur entsprechenden Spannung im
andern Netzwerk und umgekehrt.\\
Aus $\Sigma I=0$ wird $\Sigma U=0$ und umgekehrt\\
Aus Serieschaltungen werden Parallelschaltungen und umgekehrt\\
Aus Knoten werden Maschen und umgekehrt\\
\paragraph{Beispiel}
\begin{figure}[!ht]
\centering
\subfloat[Serieschaltung]{
	\tikzexternaldisable
\begin{circuitikz}[scale=2, european, american inductors]
\ctikzset{voltage/european label distance=3}
\ctikzset{bipoles/length=1.2cm}
	\draw (0,0) to[american voltage source, label=\mbox{$U_q=60V$}] (0,2)
	to [short, i=\mbox{$I=0.2$}] (2,2)
	to [R, label=\mbox{$R_1=100\Omega$}] (2,1)
	to [R, label=\mbox{$R_2=200\Omega$}] (2,0)
	-- (0,0)
	;
	\draw (1.9,2) to[open, v>=\mbox{$U_1=20V$}] (1.9,1);
	\draw (1.9,1) to[open, v>=\mbox{$U_2=40V$}] (1.9,0);
\end{circuitikz}
\tikzexternalenable

	\label{fig:elNetzwerke:Serieschaltung}
}
\qquad
\qquad
\subfloat[Parallelschaltung]{
	\begin{circuitikz}[scale=2, european, american inductors]
\ctikzset{voltage/european label distance=3}
%\ctikzset{bipoles/length=1.2cm}
	\draw (0,0)	to[american current source, label=\mbox{$I_q'=0.2V$}] (2,0)
	-- (2,-2)
	to [R, l_=\mbox{$G_2'=200S$}] (1,-2) 
	to [short, i_=\mbox{$I_2'=40A$}] (0,-2)
	-- (0,0);
	\draw (2,-1) to [R, label=\mbox{$G_1'=100S$}] (1,-1)
	to [short, i_=\mbox{$I_1'=20A$}] (0,-1);
	\draw (0,-0.12) to [open, v_<=\mbox{$U'=0.2V$}] (2,-0.12);
\end{circuitikz}

	\label{fig:elNetzwerke:Parallelschaltung} 
}
\caption[Serie- und Parallelschaltung]{Serie- und Parallelschaltung}
\label{fig:elNetzwerke}
\end{figure}

\begin{tabular}{ll}
Serieschaltung & Parallelschaltung \\
$U_1+U_2=U_q$
& $I_1'+I_2'=I_q'$\\ $U2=Uq\frac{R_2}{R_1+R_2}$ & $I_2'=I_q'\frac{G_2'}{G_1'+G_2'}$\\
\end{tabular}

\subsubsection{Dualfaktor D}
Anpassung des Impedanzlevels, z.B.\\
alle Ströme $:1000$,\\
alle Spannungen $x1000$\\
$\rightarrow$ alle Widerstände (Imped.) $x10^6$\\
$\rightarrow$ alle Leitwerte (Admitt.) $:10^6$\\
Berechnung der dualen Grössen:\\
\begin{align}
	I_q'&=\frac{U_q}{1000\Omega }\nonumber\\
	U'&=I\cdot 1000\Omega \nonumber\\
	G'&=\frac{R}{\left(1000\Omega\right)^2}\nonumber\\
	R'&=G\left(1000 \Omega\right)^2\nonumber
\end{align}
$1000 \Omega =$ Dualfaktor D, kann beliebig gewählt werden\\
\begin{tabular}{ll}
	$L \leftrightarrow C^2$ & $\underline{Y}'=\frac{\underline{Z}}{D^2}$\\
	& $j\omega C'=\frac{j \omega L}{D^2} \Rightarrow C'=\frac{L}{D^2}$\\
	& $L'=C\cdot D^2$
\end{tabular}\\
Skript Seite 37\\

Skript Seite 39: Finden der dualen Schaltung\\

\subsection{Netzwerkfunktionen}
\begin{tabular}{ll}
	\textbf{bisher:} & $j\omega$, gelegentlich abgekürzt mit $p=j\omega$ \\
	\textbf{neu:} & komplexe Frequenz $p=\sigma + j \omega$ (Erweiterung) \\
\end{tabular}
$\rightarrow$ komplexe Frequenzebene: Bild xy-Koordinatensystem, x=sigma, y=jw\\
%TODO: Bild
NB: statt p oft auch s ($\rightarrow$ Laplace Transformationen)\\
\textbf{Bedeutung:}\\
%TODO: Bild. x=re, y=im, Adach im winkel phi mit gegenuhrzeigersinnpfeil
\begin{tabular}{ll}
	\textbf{bisher:} & $a(t)=Re\{\underline{a}(t)\}=Re\{\hat{A}\cdot e^{j\phi}\cdot
	e^{j\omega t}\}$ \\
	& $\rightarrow \text{Sinussignal: } a(t)=\hat{A}\cdot\cos{\left(\omega t +
	\phi\right)}$\\ mit $p=\sigma+j\omega$ & $a(t)=Re\{\hat{A}\cdot e^{j\phi}\cdot
	e^{\sigma t}\cdot e^{j\omega t}\}$\\
	& $a(t)=\hat{A}\cdot e^{\sigma t} \cdot \cos{\left(\omega t + \phi\right)}$\\
\end{tabular}\\
Sinussignal, das exponentiell abklingt $(\sigma < 0)$ oder exponentiell
anschwillt $(\sigma > 0)$. Für $\sigma = 0$ verhält es sich wie "`früher"'.\\
\begin{itemize}
  \item Komplexe Rechnung gilt auch für $p=\sigma+j\omega$\\
  \item Wird ein lineares Netzwerk mit $a_q=\hat{A}\cdot e^{\sigma t}
  \cos\left(\omega t + \phi\right)$ angeregt, sind alle Ströme und Spannungen
  auch exponentiell abklingend oder anschwillend.
\end{itemize}
Skript Seiten 39 bis 43, Netzwerkfunktionen

\subsection{P-N-Plan, P-N-Diagramm}
\textbf{Bsp} Tiefpass 1. Ordnung\\
\begin{wrapfigure}{r}{0.5\textwidth}
\begin{circuitikz}[scale=2, european, american inductors]
\draw (0,0) to [R, l=$R$, *-*] (1,0)
	to [short, -*] (2,0);
\draw (1,0) to [C, l=$C$, *-*] (1,-1);
\draw (0,-1) to [short, *-*] (2,-1);
\draw (0,0) to[open, v>=$U_1$] (0,-1);
\draw (2,0) to[open, v^>=$U_2$] (2,-1);
\end{circuitikz}

\caption[RC-Tiefpass]{RC-Tiefpass}
\label{fig:RCTiefpass}
\end{wrapfigure}
$\underline{F}(p)=\frac{\underline{U}_2}{\underline{U}_1}=\frac{\frac{1}{pC}}{R+\frac{1}{pC}}=\frac{1}{pRC+1}
=\frac{1}{RC\left(p+\frac{1}{RC}\right)}=F_0\frac{1}{p-p_1}$\\
\begin{tabular}{ll}
Nullstellen: & keine\\
Pole: & $pRC+1=0$ \\
& $p=-\frac{1}{RC}=p_1$\\
\end{tabular}\\
%TODO Pol-Nullstellen Diagramm y=jw, x=sigma, polx bei -1/rc
$F(p)=|\underline{F}(p)|=\frac{F_0}{|p-p_1|}$\\

\subsubsection{Zusammenhang Übertragungsfunktion und freier Schwingung}
Homogene Lösung der DGL\\
%TODO Bild Rechteck "`lineares Netzwerk"', U1 x(t) am eingang, U2 y(t) am
% ausgang
\begin{align}
	\underline{F}(p)&=\frac{\underline{P}_n(p)}{\underline{Q}_m(p)}\nonumber\\
	\underline{F}(j\omega)&=\frac{P_n(j\omega)}{Q_m(j\omega)}\nonumber\\
	\underline{F}(j\omega)&=\frac{a_0+a_1j\omega+a_2(j\omega)^2+\ldots+a_n(j\omega)^n}{b_0+b_1j\omega+\ldots+b_m(j\omega)^m}=\frac{\underline{U}_2}{\underline{U}_1}\nonumber\\
	\rightarrow
\underline{U}_2(b_0+b_1j\omega+b_2(j\omega)^2+\ldots+b_m(j\omega)^m)&=\underline{U}_1(a_0+a_1j\omega+a_2(j\omega)^2+\ldots+a_n(j\omega)^n)\nonumber
\end{align}
Wir erinnern uns: Komplexe Amplitude mit $j\omega$ multiplizieren
$\widehat{=}$ ableiten.\\
\begin{align}
	&\Rightarrow U_2(t)\cdot b_0 + \dot{U}_2b_1 +
	\ddot{U}_2b_2+\ldots+U_2^{(m)}\cdot
	b_m=a_0U_1(t)+a_1\dot{U}_1+a_2\ddot{U}_1+\ldots+a_nU_1^{(n)}\nonumber\\
	&\Rightarrow b_0U_2+b_1\dot{U}_2+b_2\ddot{U}_2+\ldots+b_mU_2^{(m)}=0\nonumber
\end{align}
Ansatz $U_2=\hat{U}_2\cdot e^{\alpha t}$\\
\begin{align}
	&\Rightarrow \underbrace{\hat{U}_2e^{\alpha t}}_{\neq 0}\underbrace{(b_0+\alpha
b_1+\alpha^2b_2+\ldots+\alpha^mb_m)}_{\text{charakteristische
Lösung: }\alpha_1, \alpha_2, \ldots \alpha_m}=0\nonumber\\
%TODO 2 underbrace: udachehoch \neq 0, rest = charakteristische Lösung: Lösungen
% alpha1, alpha2, \ldots alpham
	U_{2h}&=c_1e^{\alpha_1t}+c_2e^{\alpha_2t}+\ldots+c_me^{\alpha_mt}\nonumber
\end{align}
%TODO alles in align
$\Rightarrow$ Die Lösungen $\alpha_1\ldots\alpha_m$ der charakteristischen
Gleichung entsprechen den Nullstellen von $Q_m(p)$ bzgl den Polen von
$\underline{F}(p)$\\
\begin{align}
\alpha_i&=p_i\nonumber\\
\rightarrow
U_h(t)&=\hat{U}_{h1}e^{p_1t}+\hat{U}_{h2}e^{p_2t}+\ldots+U_{hm}e^{p_mt}\nonumber
\end{align}
$Re\{p_i\} \Rightarrow $ Dämpfungskonst $=\sigma_i\ (p<0)$\\
$Im\{p_i\}=$
Eigenfrequenz\\
Skript, Seite 49\\


