\section{Elektrische Netzwerke, Systeme}
\index{Elektrische Netzwerke|see{Elektrische Systeme}}
\index{Elektrische Systeme}
Skript Herr Neubauer Seite 36-37.
Netzwerk-Klassifizierung


\subsection{Netzwerk-Klassifizierung}
\paragraph{Auflösung}
\begin{tabular}{|l|l|l|}
	\hline
	& kontinuierlich & diskret \\
	\hline
	Elemente & räumlich verteilt & konzentriert \\
	\hline
	Signal-Änderung & zeitlich stetig & intervall-bedingt\\
	& wertmässig analog & digital\\
	\hline
\end{tabular}
\paragraph{Netzwerk-Elemente}
R-, RC-, RL-, RLC-Netzwerke
\paragraph{Ein- und Ausgänge}
\begin{tabular}{llll}
Bezeichnung & Eintor=Zweipol & Zweitor=Vierpol & N-tor=2N-pol \\
Zweigpaare & 1 & 2 & N\\
\end{tabular}\\
\ldots führen aus dem elektrischen Netzwerk heraus.
\paragraph{Verwendung}
Filter, Regelsysteme, Signalformer, Nachrichtensysteme, Niederspannungsnetze
u.a.
\paragraph{Systemeigenschaften}
Aktiv – passiv, linear – nichtlinear, zeitvariant – zeitinvariant,
ohne Speicher – mit Speicher, minimalphasig – nicht minimalphasig\\


Die allgemeine Theorie der elektrischen Netzwerke begründet sich auf Arbeiten von Foster,
Cauer, Brune, Bode, Darlington, Piloty, Bader, Guillemin, Wagner, Heaviside und anderen.

\subsection{Dualität}
\subsubsection{Duale Zweipole}
$\rightarrow$ vertauschte Rollen von Spannung und Strom\\
\begin{tabular}{llll}
$u=L\frac{di}{dt}$&$ \leftrightarrow$&$ i=C\frac{du}{dt}$
& L und C sind zueinander dual\\
$u=R\cdot i $&$ \leftrightarrow $&$ i=G\cdot u$ &
R und G sind zueinander dual\\
$U_q $&$ \leftrightarrow $&$ i_q $
& Strom- und Spannungsquellen sind dual\\
\end{tabular}

\subsubsection{Duale Schaltungen}
Werden durch Gleichungen mit derselben "`mathematischen Struktur"' beschrieben.
Alle Ströme im einen Netzwerk sind proportional zur entsprechenden Spannung im
andern Netzwerk und umgekehrt.\\
Aus $\Sigma I=0$ wird $\Sigma U=0$ und umgekehrt\\
Aus Serieschaltungen werden Parallelschaltungen und umgekehrt\\
Aus Knoten werden Maschen und umgekehrt\\
\paragraph{Beispiel}
\begin{figure}[!ht]
\centering
\subfloat[Serieschaltung]{
	\tikzexternaldisable
\begin{circuitikz}[scale=2, european, american inductors]
\ctikzset{voltage/european label distance=3}
\ctikzset{bipoles/length=1.2cm}
	\draw (0,0) to[american voltage source, label=\mbox{$U_q=60V$}] (0,2)
	to [short, i=\mbox{$I=0.2$}] (2,2)
	to [R, label=\mbox{$R_1=100\Omega$}] (2,1)
	to [R, label=\mbox{$R_2=200\Omega$}] (2,0)
	-- (0,0)
	;
	\draw (1.9,2) to[open, v>=\mbox{$U_1=20V$}] (1.9,1);
	\draw (1.9,1) to[open, v>=\mbox{$U_2=40V$}] (1.9,0);
\end{circuitikz}
\tikzexternalenable

	\label{fig:elNetzwerke:Serieschaltung}
}
\qquad
\qquad
\subfloat[Parallelschaltung]{
	\begin{circuitikz}[scale=2, european, american inductors]
\ctikzset{voltage/european label distance=3}
%\ctikzset{bipoles/length=1.2cm}
	\draw (0,0)	to[american current source, label=\mbox{$I_q'=0.2V$}] (2,0)
	-- (2,-2)
	to [R, l_=\mbox{$G_2'=200S$}] (1,-2) 
	to [short, i_=\mbox{$I_2'=40A$}] (0,-2)
	-- (0,0);
	\draw (2,-1) to [R, label=\mbox{$G_1'=100S$}] (1,-1)
	to [short, i_=\mbox{$I_1'=20A$}] (0,-1);
	\draw (0,-0.12) to [open, v_<=\mbox{$U'=0.2V$}] (2,-0.12);
\end{circuitikz}

	\label{fig:elNetzwerke:Parallelschaltung} 
}
\caption[Serie- und Parallelschaltung]{Serie- und Parallelschaltung}
\label{fig:elNetzwerke}
\end{figure}

\begin{tabular}{ll}
Serieschaltung & Parallelschaltung \\
$U_1+U_2=U_q$
& $I_1'+I_2'=I_q'$\\ $U2=Uq\frac{R_2}{R_1+R_2}$ & $I_2'=I_q'\frac{G_2'}{G_1'+G_2'}$\\
\end{tabular}

\subsubsection{Dualfaktor D}
Anpassung des Impedanzlevels, z.B.\\
alle Ströme $:1000$,\\
alle Spannungen $x1000$\\
$\rightarrow$ alle Widerstände (Imped.) $x10^6$\\
$\rightarrow$ alle Leitwerte (Admitt.) $:10^6$\\
Berechnung der dualen Grössen:\\
\begin{align}
	I_q'&=\frac{U_q}{1000\Omega }\nonumber\\
	U'&=I\cdot 1000\Omega \nonumber\\
	G'&=\frac{R}{\left(1000\Omega\right)^2}\nonumber\\
	R'&=G\left(1000 \Omega\right)^2\nonumber
\end{align}
$1000 \Omega =$ Dualfaktor D, kann beliebig gewählt werden\\
\begin{tabular}{ll}
	$L \leftrightarrow C^2$ & $\underline{Y}'=\frac{\underline{Z}}{D^2}$\\
	& $j\omega C'=\frac{j \omega L}{D^2} \Rightarrow C'=\frac{L}{D^2}$\\
	& $L'=C\cdot D^2$
\end{tabular}\\
Skript Seite 37
