\subsection{Freie Schwingung des Serieschwingkreises}
\index{Schwingkreis!seriell}
\begin{wrapfigure}{l}{0.5\textwidth}
	\centering
	\begin{tikzpicture}[circuit ee IEC, x=2cm, y=2cm, semithick]
	
	
	\node (start) [contact] at (0,0) {};
	\node (end) [contact] at (0,-3) {};
	
	\draw (start) to [inductor={info=L, name=L}] (0,-1)
					to [resistor={info=R, name=R}] (0,-2)
					to [capacitor={info=C, name=C}] (0,-3)
					-- (end);
	\draw (start) -- (1,0);
	\draw (end) -- (1,-3);
	\draw (1,0) -- (1,-3);
	
\end{tikzpicture}
	\caption{Serieschwingkreis}
	\label{fig:SerieSKGeschlossen}
\end{wrapfigure}

\textbf{gesucht:  }$i(t)$\\
\textbf{gegeben:}
\begin{align}
	u_L + u_R + u_C &= 0\nonumber\\
	L\dot{i} + Ri + \frac{1}{C}\int{i}dt&=0\nonumber\\
	\boxed{\ddot{i}+\frac{R}{L}\dot{i}+\frac{1}{LC}i=0}
\end{align}\\
Gleiche Dgl wie bei Parallel-SK. mit folgenden Entsprechungen:\\
\begin{align}
u &\leftrightarrow i\nonumber\\
L &\leftrightarrow C\nonumber\\
R &\leftrightarrow G=\frac{1}{R}\nonumber
\end{align}
Dualität:\\
\begin{align}
\omega_r&=\frac{1}{\sqrt{LC}}\nonumber\\
Q_s=\frac{1}{R}\sqrt{\frac{L}{C}}\ (vgl:\ Q_P&=R\sqrt{\frac{C}{L}})\nonumber\\
\boxed{\ddot i + \frac{\omega_r}{Q_s}\dot i + \omega_r^2\cdot i = 0}
\end{align}\\

