\subsection{Erzwungene Schwingung}
\index{schwingung!erzwungen}
Vorab:\\ %Grafik einfügen Iq-l-r-c parallel, il,ir,ic angeschrieben
Resonanz $\omega=\omega_r$\\
$\underline{Y}_C +\underline{Y}_L = 0$\\
$j\omega C+\frac{1}{j\omega L} = 0$\\
$\omega=\omega_r=\frac{1}{\sqrt{LC}}$\\
%Diagramm einfügen
%leider wird das nicht möglich sein da Damian ein elender fussmaler ist!
$I_C=U\cdot\omega_rC=I\cdot R \cdot \omega_rC=I\cdot R \frac{C}{\sqrt{LC}}$\\
$=I\cdot R \sqrt{\frac{C}{L}} = I \cdot Q$\\
$I_L=\frac{U}{\omega_rL}=I \cdot Q$\\

%Uq-L-R-C seriell, ul uc i
$\omega_r=\frac{1}{\sqrt{LC}}$\\
%nippelgraph 1
$\frac{f_r}{\delta f}=\frac{\omega_r}{\delta \omega} = Q$

\subsubsection{Parallelschwingkreis}
$\underline{Z} = \frac{1}{\frac{1}{j\omega L}+\frac{1}{R} + j \omega C}
= \frac{j\omega L}{1+j\omega \frac{L}{R}+(j\omega)^2LC}
= \frac{R}{1+(\frac{1}{j\omega L}+j\omega C)R}$\\
normiert:\\
$\frac{\underline{Z}}{R}=\frac{1}{1+j(\omega C-\frac{1}{\omega L})\cdot R}
=\frac{j\omega \frac{L}{R}}{1+j\omega\frac{L}{R}+(j\omega)^2LC}
=\frac{1}{\sqrt{1+(\omega C - \frac{1}{\omega
L})^2R^2}} winkel \arctan{((\frac{1}{\omega L}-\omega C)\cdot R)} $\\

Frequenzgang von $\underline Z$
Spez. Pt. $\Im{\{Z\}} = 0 \rightarrow \omega = \omega_r=\frac{1}{\sqrt{LC}}$\\
$Z = Z_{max} = R$
Tiefe Frequenzen $\omega \ll \omega_r: \frac{\underline Z}{R} \approx
j\omega\frac{L}{R}=\omega\frac{L}{R} winkel +90grad induktiv$\\
$\omega \gg \omega_r: \frac{\underline Z}{R} \approx \frac{1}{j\omega RC}
winkel -90 grad kapazitiv$\\
3dB-Pt\\
%nippelgraph 2
$\frac{Z}{R}=\frac{1}{\sqrt{2}}$\\
$\frac{1}{1+(\omega C-\frac{1}{\omega L})^2R^2}=\frac{1}{2}$\\
$(\omega C-\frac{1}{\omega L})^2R^2 = 1$\\
$(\omega C - \frac{1}{\omega L})R = \pm 1$
$\omega C - \frac{1}{\omega L} = \pm \frac{1}{R}$\\
$\omega^2 \pm \omega \frac{1}{RC}-\frac{1}{LC} = 0$\\
$\omega^2 + \frac{\omega_r}{Q}\cdot \omega - \omega_r^2=0 bzw \omega^2 -
\frac{\omega_r}{Q}\cdot \omega - \omega_r^2=0$\\
$\omega_{1,1,2} = -\frac{\omega_r}{2Q} \pm \omega_r\sqrt{\frac{1}{4Q^2}+1} bzw
+\frac{\omega_r}{2Q} \pm \omega_r\sqrt{\frac{1}{4Q^2}+1}$
%geschwoffene klammern unten dran mechen //TODO
3dB Frequenzgrenzen: $\omega_{1,2}=\omega_r(\sqrt{\frac{1}{4Q^2}+1}\mp\frac{1}{2Q})$
