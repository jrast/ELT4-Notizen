\subsection{Erzwungene Schwingung}
\index{Schwingung!erzwungene}

Eine erzwungene Schwingung tritt ein, sobald der Schwinkreis mit einer sinus-
oder co"-sin"-us"-för"-migen Quelle angeregt wird. Auf der untenstehenden
Abbildung (Abb. \ref{fig:schwingkreise:erzwungene}) ist dies für den Parallelschwingkreis
dargestellt. Zudem ist auf der Abbildung
\ref{fig:schwingkreise:erzwungener:sinus} der Spannungs- und Stromverlauf
ersichtlich.

\begin{figure}[!ht]
\centering
\subfloat[Erzwungener Parallel-Schwingkreis]{
	\tikzexternaldisable
\begin{circuitikz}[scale=2, european, american inductors, american currents]
\ctikzset{voltage/european label distance=2}
\ctikzset{bipoles/length=1.2cm}
	\draw
	(0,0)
		to[short](1,0)
		to[short, *-](2,0)
		to[short, *-](3,0)
	(0,0) to[isourcesin, i=$I$] (0,-1)
	(1,0) to[L=L, i>^=$i_L$](1,-1)
	(2,0) to[R=R, i>^=$i_R$](2,-1)
	(3,0) to[C=C, i>^=$i_C$](3,-1)
		to[short, -*](2,-1)
		to[short, -*](1,-1)
		to[short](0,-1);
\end{circuitikz}
\tikzexternalenable
	\label{fig:schwingkreise:erzwungener:parallel}
}
\qquad
\subfloat[Strom und Spannung im Schwingkreis]{
	\begin{tikzpicture}[domain=0:6]
	
	\draw[->] (-0.2,0) -- (6.2,0) node[right] {};
	\draw[->] (0,-1.2) -- (0,1.2) node[above] {};
	\draw[color=blue, samples=150]   plot (\x,{0.5*sin(\x r)})   node[right]
	{$i(t)=i_R$};
	\draw[color=blue, samples=150]   plot (\x,{0.75*sin(\x r)})   node[below right]
	{$u(t)=i\cdot R$};
	\draw[color=green, samples=150]   plot (\x,{cos(\x r)})   node[right]
	{$i_C(t)$};
	\draw[color=red, samples=150]   plot (\x,{-1*cos(\x r)})  
	node[right] {$i_L(t)$};
\end{tikzpicture} 
	\label{fig:schwingkreise:erzwungener:sinus} 
}
\caption{Erzwungener Schwingkreis}
\label{fig:schwingkreise:erzwungene}
\end{figure}


\subsubsection{Resonanz $\omega = \omega_r$}
\index{Resonanz}
Resonanz tritt ein, wenn sich die Impedanzen der Kapazität und der Induktivität
gegenseitig aufheben:
\begin{align}
	\underline{Y}_C +\underline{Y}_L &= 0\nonumber \\
	j\omega C+\frac{1}{j\omega L} &= 0\nonumber \\
	\omega=\omega_r &= \frac{1}{\sqrt{LC}}\nonumber
\end{align}

\begin{figure}[!ht]
	\centering
	\pgfmathdeclarefunction{gauss}{2}{%
  \pgfmathparse{1/(#2*sqrt(2*pi))*exp(-((x-#1)^2)/(2*#2^2))}%
}

\makebox{\begin{tikzpicture}[scale=0.7, every node/.style={scale=1.3}]
\begin{axis}[
  no markers,
  domain=0:6,
  samples=500,
  hide x axis=true,
	hide y axis=true,
  %height=5cm, width=12cm,
  %xtick={4,6.5}, ytick=\empty,
  enlargelimits=false,
  clip=false,
  axis on top,
  grid = major
  ]
  \addplot {gauss(3,1)};
\end{axis}

	\draw[->, thick] (-0.2,0) -- +(right:7.5) node[right] {$t$};
	\draw (3.43,0) node[anchor=north] {$\omega_r$};
		\draw[dashed] (3.43,0) -- +(0,6);
	\draw (2.43,0) node[anchor=north] {$\omega_1$};
	\draw (4.43,0) node[anchor=north] {$\omega_2$};

	\draw[->, thick] (0,-0.2) -- +(north:6.5) node[above] {$\underline{Z}$};
	\draw (0.05,3.85) -- (-0.05,3.85) node[anchor=east] {$\frac{R}{\sqrt{2}}$};
	\draw (0.05,5.7) -- (-0.05,5.7) node[anchor=east] {$R$};

	\draw[dashed] (0,3.85) -- (4.5,3.85);
	\draw[dashed] (0,5.7) -- (3.55,5.7);
	\draw[dashed, shorten <= -3pt,] (2.43,4) -- (2.43,0);
	%\draw[dashed, shorten <= -3pt] (3.43,5.75) -- (3.43,0);
	\draw[dashed, shorten <= -3pt] (4.43,4) -- (4.43,0);
	\draw[<->] (4.5,4) -- (4.5,5.75) node[below right] {$3dB$};


\end{tikzpicture}}

	\caption{Frequenz-Impedanz-Diagramm mit 3dB Punkten}
	\label{fig:fZDiagramm}
\end{figure}

Wird der Schwingkreis mit seiner Resonanzfrequenz angeregt, steigen die Ströme
in den Bauteilen erheblich an und werden nur durch die Innenwiederstände
begrenzt. Die Ströme werden also mit der Güte $Q$ multipliziert:

\begin{align}
	I_C&=U\cdot\omega_rC = I\cdot R \cdot \omega_rC = I\cdot R
	\frac{C}{\sqrt{LC}} \nonumber\\
	&=I\cdot R \sqrt{\frac{C}{L}} = I \cdot Q\nonumber\\
	I_L&=\frac{U}{\omega_rL}=I \cdot Q\nonumber
\end{align}
Die Punkte $\omega_{1,2}$ in der obigen Abbildung zeigen die Punkte, an welchen
die Verstärkung 3dB unter dem Maximum liegt. Anhand dieser Werte und den
zugehörignen Z-Werten kann man auch daraus die Güte des Schwingkreises
berechnen:
\begin{align}
	\frac{f_r}{\Delta f}=\frac{\omega_r}{\Delta \omega} &= Q\nonumber
\end{align}
wobei folgendes gilt: $\Delta \omega = \omega_2-\omega_1 $

  
\subsubsection{Parallelschwingkreis}
\index{Schwingung!erzwungene}
$$\underline{Z} = \frac{1}{\frac{1}{j\omega L}+\frac{1}{R} + j \omega C}
= \frac{j\omega L}{1+j\omega \frac{L}{R}+(j\omega)^2LC}
= \frac{R}{1+(\frac{1}{j\omega L}+j\omega C)R}$$\\
normiert:
\begin{align}
\frac{\underline{Z}}{R}&=\frac{1}{1+j(\omega C-\frac{1}{\omega L})\cdot R}
=\frac{j\omega \frac{L}{R}}{1+j\omega\frac{L}{R}+(j\omega)^2LC}\nonumber\\
&=\frac{1}{\sqrt{1+(\omega C - \frac{1}{\omega
L})^2R^2}} \angle \arctan{\left(\left(\frac{1}{\omega L}-\omega C\right)\cdot
R\right)}
\nonumber
\end{align}


\paragraph{Frequenzgang von $\underline Z$}
\index{Frequenzgang!Parallelschwingkreis}
\begin{description}
[style=multiline,topsep=0pt,leftmargin=4.5cm,rightmargin=2cm]
  \item[Spezieller Punkt]
  	$\operatorname{Im}{\{Z\}} = 0 \rightarrow \omega =
	\omega_r=\frac{1}{\sqrt{LC}}$ \newline
	$Z = Z_{max} = R$
  \item[Tiefe Frequenzen]
    $\omega \ll \omega_r: \frac{\underline Z}{R} \approx
	j\omega\frac{L}{R}=\omega\frac{L}{R} \angle +90^\circ$ induktiv \newline
	$\omega \gg \omega_r: \frac{\underline Z}{R} \approx
	  \frac{1}{j\omega RC} \angle -90^\circ$ kapazitiv
  \item[3dB-Punkt]
    $\frac{Z}{R}=\frac{1}{\sqrt{2}}$ \newline
	$\frac{1}{1+(\omega C-\frac{1}{\omega L})^2R^2}=\frac{1}{2}$
\end{description}


% \begin{wrapfigure}{l}{0.5\textwidth}
% 	\centering
% 	\pgfmathdeclarefunction{gauss}{2}{%
  \pgfmathparse{1/(#2*sqrt(2*pi))*exp(-((x-#1)^2)/(2*#2^2))}%
}

\makebox{\begin{tikzpicture}[scale=0.7, every node/.style={scale=1.3}]
\begin{axis}[
  no markers,
  domain=0:6,
  samples=500,
  hide x axis=true,
	hide y axis=true,
  %height=5cm, width=12cm,
  %xtick={4,6.5}, ytick=\empty,
  enlargelimits=false,
  clip=false,
  axis on top,
  grid = major
  ]
  \addplot {gauss(3,1)};
\end{axis}

	\draw[->, thick] (-0.2,0) -- +(right:7.5) node[right] {$t$};
	\draw (3.43,0) node[anchor=north] {$\omega_r$};
		\draw[dashed] (3.43,0) -- +(0,6);
	\draw (2.43,0) node[anchor=north] {$\omega_1$};
	\draw (4.43,0) node[anchor=north] {$\omega_2$};

	\draw[->, thick] (0,-0.2) -- +(north:6.5) node[above] {$\underline{Z}$};
	\draw (0.05,3.85) -- (-0.05,3.85) node[anchor=east] {$\frac{R}{\sqrt{2}}$};
	\draw (0.05,5.7) -- (-0.05,5.7) node[anchor=east] {$R$};

	\draw[dashed] (0,3.85) -- (4.5,3.85);
	\draw[dashed] (0,5.7) -- (3.55,5.7);
	\draw[dashed, shorten <= -3pt,] (2.43,4) -- (2.43,0);
	%\draw[dashed, shorten <= -3pt] (3.43,5.75) -- (3.43,0);
	\draw[dashed, shorten <= -3pt] (4.43,4) -- (4.43,0);
	\draw[<->] (4.5,4) -- (4.5,5.75) node[below right] {$3dB$};


\end{tikzpicture}}

% 	\caption{3dB-Punkt}
% 	\label{fig:3dBPunkt}
% \end{wrapfigure}


\begin{align}
	\left(\omega C-\frac{1}{\omega L}\right)^2R^2 &= 1\nonumber\\
	\left(\omega C - \frac{1}{\omega L}\right)R &= \pm 1\nonumber\\
	\omega C - \frac{1}{\omega L} &= \pm \frac{1}{R}\nonumber\\
	\omega^2 \pm \omega \frac{1}{RC}-\frac{1}{LC} &= 0\nonumber\\
	\omega^2 + \frac{\omega_r}{Q}\cdot \omega - \omega_r^2&=0 \\ bzw\\ \omega^2 -
	\frac{\omega_r}{Q}\cdot \omega - \omega_r^2=0\nonumber\\
	\omega_{1,1,2} = -\frac{\omega_r}{2Q} \pm 
	\underbrace{\omega_r\sqrt{\frac{1}{4Q^2}+1}}_{>\frac{\omega_r}{2Q}}
	\\	bzw \\
	\omega_{2,1,2}=\frac{\omega_r}{2Q} \pm
	\underbrace{\omega_r\sqrt{\frac{1}{4Q^2}+1}}_{>\frac{\omega_r}{2Q}} \nonumber\\
\end{align}

3dB Frequenzgrenzen:
\begin{align}
	\boxed{\omega_{1,2}=\omega_r\left(\sqrt{\frac{1}{4Q^2}+1}\mp\frac{1}{2Q}\right)}\\
	\approx \omega_r\left(1\mp\frac{1}{2Q}\right) \text{ für } Q \gg 1\nonumber\\
	\Rightarrow \omega_1\cdot\omega_2=\omega_r^2 (\omega_r \text{ geom Mittel von
		} \omega_1 \text{ und } \omega_2)\nonumber\\
	\omega_2 - \omega_1 = \frac{\omega_r}{Q} = b_{\omega}\nonumber\\
	\boxed{\text{Bandbreite } B = f_2 - f_1 = \frac{f_r}{Q}}\\
	\text{Frequenzabstand der 3dB Punkte}\nonumber
\end{align}

\begin{figure}[!ht]
	\centering
	\pgfmathdeclarefunction{ImpedanzBetrag}{3}{%
  \pgfmathparse{#1/sqrt(1+(x*#2-1/(x*#3))^2*#1^2)}%
}
	%\pgfmathparse{R/sqrt(1+(xC-1/xL)^2*R^2)}
	%R=1000
	%C=1*10^(-6)=0.000001
	%L=10*10^-(3)=0.01

\makebox{\begin{tikzpicture}[scale=2, every node/.style={scale=0.6}]
\begin{axis}[
  no markers,
  domain=5*10^3:15*10^3,
  %hide x axis=true,
  %hide y axis=true,
  samples=500,
  %line width = 1.5pt,
  axis x line*=bottom,
  axis y line*=left,
  xtick={4000,5000,...,15000,16000},
  ytick={0,100,200,...,1000},
  ymin=0,
  xmin=4000,
  xmax=16000,
  enlargelimits=false,
  clip=false,
  %axis on top,
  grid = major,
  xlabel={Kreisfrequenz $\omega[s^{-1}]$},
  ylabel={Impedanz Betrag $Z[\Omega]$}
  ]
  \addplot{ImpedanzBetrag(1000, 0.000001, 0.01)};
  \addplot[dashed] plot coordinates  {
  	(9000,707)
  	(16000,707)
  };
  
  \addplot[dashed] plot coordinates  {
  	(9500,780)
  	(9500,0)
  };
  \addplot[dashed] plot coordinates  {
  	(10500,780)
  	(10500,0)
  };
  \node[coordinate,label=right:{$1/\sqrt{2}$}]
    at (axis cs:16000,707) {};
  \node[coordinate,label=below:{$\omega_1$}]
    at (axis cs:9500,-30) {};
  \node[coordinate,label=below:{$\omega_2$}]
    at (axis cs:10500,-30) {};
\end{axis}
\end{tikzpicture}}



	\caption{Frequenzgang}
	\label{fig:ParallelImpedanzBetrag}
\end{figure}
\begin{figure}[!ht]
	\centering
	\pgfmathdeclarefunction{ImpedanzPhase}{3}{%
  \pgfmathparse{atan((1/(x*#3)-x*#2)*#1)}%
}

\makebox{\begin{tikzpicture}[scale=2, every node/.style={scale=0.6}]
\begin{axis}[
  no markers,
  domain=5*10^3:15*10^3,
  samples=500,
  axis x line*=bottom,
  axis y line*=left,
  xtick={4000,5000,...,15000,16000},
  ytick={-90,-80,...,90},
  ymin=-90,
  ymax=90,
  xmin=4000,
  xmax=16000,
  enlargelimits=false,
  clip=false,
  axis on top,
  grid = major,
  xlabel={Kreisfrequenz $\omega[s^{-1}]$},
  ylabel={Impedanz Phase $\phi[^{\circ}]$}
  ]
  \addplot{ImpedanzPhase(1000, 0.000001, 0.01)};
  \addplot[dashed] plot coordinates  {
  	(4000,45)
  	(9700,45)
  };
  \addplot[dashed] plot coordinates  {
  	(4000,-45)
  	(10700,-45)
  };
  \addplot[dashed] plot coordinates {
  	(9500,90)
  	(9500,35)
  };
  \addplot[dashed] plot coordinates {
  	(10500,90)
  	(10500,-55)
  };
  \node[coordinate,label=right:{$45^{\circ}$}]
  	at (axis cs:9650,45) {};
  \node[coordinate,label=right:{$-45^{\circ}$}]
  	at (axis cs:10650,-45) {};
\end{axis}
\end{tikzpicture}}



	\caption{Phasengang}
	\label{fig:ParallelImpedanzPhase}
\end{figure}

\subsubsection{Frequenzgang von Spannung und Strömen}
\index{Frequenzgang!Spannung}
\index{Frequenzgang!Strom}
\begin{align}
	\underline{U} &= \underline{Z} \cdot \underline{I},
	&& \text{normiert: }
	&& \frac{\underline{U}}{\underline{U_{max}}} =
	\frac{\underline{Z}}{\underline{Z_{max}}} = \frac{\underline{Z}}{R}\nonumber\\
	\underline{I_R} &= \frac{\underline{U}}{R},
	&& \text{normiert:}
	&& \frac{\underline{I_R}}{\underline{I}} = \frac{\underline{Z}}{R}\nonumber\\
	\underline{I_L} &= \frac{\underline{U}}{j\omega L} =
	\frac{\underline{I}\cdot\underline{Z}}{j\omega L},
	&& \text{normiert:	}
	&& \frac{\underline{I_L}}{\underline{I}}=\frac{\underline{Z}}{j\omega
	L}\nonumber\\
	\underline{I_C} &= j\omega C \underline{U},
	&& \text{normiert:}
	&& \frac{\underline{I_C}}{\underline{I}} = j\omega C\cdot Z\nonumber
\end{align}

\paragraph{Frequenzgang von $I_L, I_C$}
\index{Frequenzgang!Strom}
Fall $\omega \ll \omega_r (Z \approx \omega L)$:
\begin{align}
	\frac{I_L}{I} &\approx 1\nonumber\\
	\frac{I_C}{I} &\approx \omega^2LC\nonumber
\end{align}
Fall $\omega \gg \omega_r (Z \approx \frac{1}{\omega C})$:
\begin{align}
	\frac{I_L}{I} &\approx \frac{1}{\omega^2LC}\nonumber\\
	\frac{I_C}{I} &\approx 1\nonumber
\end{align}
Fall $\omega = \omega_r$:
\begin{align}
	\frac{I_L}{I} &= \frac{Z(\omega_R)}{\omega_r L} =
	\frac{R}{\frac{1}{\sqrt(LC)}\cdot L} = \frac{R\sqrt{L}\sqrt{C}}{L} =
	R\cdot\sqrt{\frac{C}{L}} = Q\nonumber\\
	\frac{I_C}{I} &= Z(\omega_r)\cdot \omega_r C = R \frac{C}{\sqrt{LC}} = R\cdot
	\sqrt{\frac{C}{L}} = Q\nonumber
\end{align}

Bei $Q>1$: Stromüberhöhung in $L$ und $C$ bei Resonanz!

%Neubauer Skript Seite 17 einfügen
%Für Formäln Böööörni fragen
% \begin{figure}[!h]
% 	\centering
% 	\pgfmathdeclarefunction{StromBetragL}{3}{%
	\pgfmathparse{#1/sqrt((x*#3)^2+((x^2*#2*#3)-1)^2(#1)^2)}%
	%\pgfmathparse{1000/sqrt((x*0.01)^2+((x^2*0.000001*0.01)-1)^2(1000)^2)}%
}

%Teständerung


\makebox{\begin{tikzpicture}[every node/.style={scale=0.6}]
\begin{axis}[
	domain=5*10^3:15*10^3,
	axis x line*=bottom,
  axis y line*=left,
  xtick={4000,5000,...,15000,16000},
  ytick={0,100,200,...,1000},
  ymin=0,
  xmin=4000,
  xmax=16000,
  samples=500,
  grid = major,
  xlabel={Kreisfrequenz $\omega[s^{-1}]$},
  ylabel={Strom Betraege}
]
\addplot{StromBetragL(1000, 0.000001, 0.01)};
\end{axis}
\end{tikzpicture}}

% \pgfmathdeclarefunction{ImpedanzBetrag}{3}{%
%   \pgfmathparse{#1/sqrt(1+(x*#2-1/(x*#3))^2*#1^2)}%
% }
% \pgfmathdeclarefunction{ImpedanzStromL}{3}{%
% 	\pgfmathparse{#1}%
% }
% 	%\pgfmathparse{R/sqrt(1+(xC-1/xL)^2*R^2)}
% 	%R=1000
% 	%C=1*10^(-6)=0.000001
% 	%L=10*10^-(3)=0.01
% 
% \makebox{\begin{tikzpicture}[scale=2, every node/.style={scale=0.6}]
% \begin{axis}[
%   no markers,
%   domain=5*10^3:15*10^3,
%   %hide x axis=true,
%   %hide y axis=true,
%   samples=500,
%   %line width = 1.5pt,
%   axis x line*=bottom,
%   axis y line*=left,
%   xtick={4000,5000,...,15000,16000},
%   ytick={0,100,200,...,1000},
%   ymin=0,
%   xmin=4000,
%   xmax=16000,
%   enlargelimits=false,
%   clip=false,
%   axis on top,
%   grid = major,
%   xlabel={Kreisfrequenz $\omega[s^{-1}]$},
%   ylabel={Impedanz Betrag $Z[\Omega]$}
%   ]
%   %\addplot{ImpedanzBetrag(1000, 0.000001, 0.01)};
% %   \addplot[dashed] plot coordinates  {
% %   	(9000,707)
% %   	(16000,707)
% %   };
% %   
% %   \addplot[dashed] plot coordinates  {
% %   	(9500,780)
% %   	(9500,0)
% %   };
% %   \addplot[dashed] plot coordinates  {
% %   	(10500,780)
% %   	(10500,0)
% %   };
% %   \node[coordinate,label=right:{$1/\sqrt{2}$}]
% %     at (axis cs:16000,707) {};
% %   \node[coordinate,label=below:{$\omega_1$}]
% %     at (axis cs:9500,-30) {};
% %   \node[coordinate,label=below:{$\omega_2$}]
% %     at (axis cs:10500,-30) {};
% \end{axis}
% \end{tikzpicture}}
% 
% 

% 	\caption{Frequenzgang der Ströme}
% 	\label{fig:ParallelStromBetrag}
% \end{figure}

\paragraph{Normierter Frequenzgang}
\index{Frequenzgang!normierter}
1. Normierung: $\omega_r=\frac{1}{\sqrt{LC}}$ als Bezug
\begin{align}
	\frac{\underline{Z}}{R} &=
	\frac{j\omega\frac{L}{R}}{1+j\omega\frac{L}{R}+(j\omega)^2LC}\nonumber
\end{align}
Mit:
\begin{align}
	LC&=\frac{1}{\omega_r^2}\nonumber\\
	\omega_rQ &= \frac{R}{L} \Rightarrow \frac{L}{R}=\frac{1}{\omega_rQ}\nonumber\\
	\frac{\underline{Z}}{R} &=
	\frac{j\frac{\omega}{\omega_r}\frac{1}{Q}}{1+j\frac{\omega}{\omega_r}\frac{1}{Q}+\left(j\frac{\omega}{\omega_r}\right)^2}\nonumber\\
	&=\frac{1}{1+\frac{1}{j\frac{\omega}{\omega_r}\frac{1}{Q}}+j\frac{\omega}{\omega_r}Q}\nonumber
	\end{align}
	\begin{align}
	\boxed{\frac{\underline{Z}}{R} =
	\frac{1}{1+jQ\left(\frac{\omega}{\omega_r}-\frac{\omega_r}{\omega}\right)} =
	F\left(\frac{\omega}{\omega_r}\right)}
\end{align}
2. Normierung: Verstimmung $\eta$ (eigentlich $\nu$)
\begin{align}
	\boxed{\eta=\frac{\omega}{\omega_r}-\frac{\omega_r}{\omega}} = \frac{f}{f_r} -
	\frac{f_r}{f}
\end{align}\\
\begin{tabular}{l|l|l|l|l|l|l|l|l|}
	$\omega =$ &
	$\omega_r$ &
	$>\omega_r$ &
	$<\omega_r$ &
	$0$ &
	$\inf$ &
	$\omega_1(3dB)$ &
	$\omega_2(3dB)$ &
	$\approx\omega_r \text{ Abweichung: } \Delta\omega$	
	\\
	\hline
	$\eta =$ &
	$0$ &
	positiv &
	negativ &
	$-\inf$ &
	$+\inf$ &
	$-1$ &
	$+1$ &
	$\approx\frac{2\cdot\Delta\omega}{\omega_r}$\\	
\end{tabular}

$\rightarrow$ Amplituden- und Phasengang werden symmetrisch
\begin{align}
 	\frac{\underline{Z}}{R} &= \frac{1}{1+j\eta Q},\ \Omega=\eta Q
 	&&\text{normierte Frequenz}\nonumber\\
 	\frac{\underline{Z}}{R} &= \frac{1}{1+j\Omega} =
 	\frac{1}{\sqrt{1+\Omega^2}}\angle
 	\arctan{\Omega} &&\text{normierter
 	Frequenzgang}\nonumber
\end{align}

\paragraph{Anwendungsbeispiel 1}
\textbf{Gegeben:} $f_r, Q$\\
\textbf{Gesucht:} Wert der Resonanzkurve bei bestimmter Frequenz $f_x$\\
\textbf{Vorgehen:}
\begin{align}
	\eta_x &= \frac{f_x}{f_r}-\frac{f_r}{f_x}\nonumber\\
	\left| F_x \right| &= \frac{1}{\sqrt{1+(\eta Q)^2}}\nonumber
\end{align}
\paragraph{Anwendungsbeispiel 2}
\textbf{Gegeben:} $f_r$, Abfall $\left|F_x\right|$ bei Frequenz $f_x$\\
\textbf{Gesucht:} $\eta$ und $Q$\\
\textbf{Vorgehen:} \\
Mit dem folgenden Ansatz kommt man durch umformen auf die gesuchte Lösung. 
\begin{align}
	\frac{1}{\sqrt{1+(\eta Q)^2}} &= \left| F_x \right|\nonumber\\
	\eta &= \frac{f_x}{f_r}-\frac{f_r}{f_x}\nonumber\\
	Q &= \frac{\eta Q}{\eta}\nonumber
\end{align}
%Seite 23 Skript Nüübuur

\subsubsection{Serieschwingkreis}
\index{Schwingung!erzwungene}
\begin{wrapfigure}{l}{0.55\textwidth}
\centering
	\tikzexternaldisable
\begin{circuitikz}[scale=2, european resistors, american inductors]
\ctikzset{voltage/european label distance=3}
\ctikzset{bipoles/length=1.2cm}
	\draw (0,0)
		to[short, i>^=$i(t)$] (1,0)
		to[L=L] (2,0)
		to[R=G] (3,0)
		to[C=C] (4,0)
		-- (4,-1.2) 
		to[vsourcesin] (0,-1.2) -- (0,0)
		%Pfeile
		(1,-0.1) to[open, v>=$\underline{U_L}$] (2,-0.1) 
		(2,-0.1) to[open, v>=$\underline{U_R}$] (3,-0.1)
		(3.2,-0.1) to[open, v>=$\underline{U_C}$] (3.8,-0.1)
		(2.5,-1) to[open, v>=$U$] (1.5,-1);
\end{circuitikz}
\tikzexternalenable
	\vspace{-0.15cm}
	\caption{Erzwungener Serieschwingkreis}
	\vspace{-1.0cm}
	\label{fig:SerieSKErzwungen}
\end{wrapfigure}

Der Serienschwingkreis lässt sich gut mit dem Parallelschwingkreis vergleichen.
Alle Grössen sind "`dual"', lassen sich also über eine einfache Umformung mit
den Grössen des Parallelschwingkreises vergleichen.
\begin{align}
  \underline{Y} &= \frac{1}{\frac{1}{G}+j\omega L+\frac{1}{j\omega C}} \nonumber
  \\
  &= \frac{G}{1+j(\omega L-\frac{1}{\omega C})G} \nonumber
\end{align}

Dabei gelten folgende Beziehungen zum Parallelschwingkreis:
\begin{align}
\underline{Z} &\leftrightarrow \underline{Y} && L \leftrightarrow C\nonumber\\
R &\leftrightarrow G && \underline{U} \leftrightarrow \underline{I}\nonumber\\
\underline{I_R} &\leftrightarrow \underline{U_R} && \underline{I_L}
\leftrightarrow \underline{U_C}\nonumber\\
\underline{I_C} &\leftrightarrow \underline{U_L}\nonumber
\end{align}

Im folgenden noch ein Beispiel, welches den Zusammenhang verdeutlicht:
\begin{align}
\text{Parallel-SK} && Q_P&=R\sqrt{\frac{C}{L}}\nonumber\\
\text{Serie-SK} &&
Q_S&=G\sqrt{\frac{L}{C}}=\frac{1}{R}\sqrt{\frac{L}{C}}\nonumber
\end{align}

\subsubsection{Serieschaltung $\leftrightarrow$ Parallelschaltung}
\index{Schaltung!Serie}
\index{Schaltung!Parallel}
Skript von Herr Neubauer