\section{Leitungstheorie}
\index{Leitungstheorie}
Ebook Strauss\\
Zur Einführung:\\
Bild 1\\
\begin{align}
	\Delta U &= -R^\prime\cdot\Delta Z \cdot i - L^\prime\Delta Z \cdot \dot
	i\nonumber\\
	\frac{\Delta U}{\Delta Z} &= \left. -R^\prime\cdot i - L^\prime\dot i \right|
	\Delta Z \rightarrow 0\nonumber\\
	\boxed{\frac{\delta u}{\delta z}=-R^\prime i- L^\prime \frac{\delta i}{\delta
	t}}\label{leitungstheorie:wellengleichung1}
\end{align}
Korrektur Buch Seite 61, Formel 3.3, Untere Formel u durch i ersetzen\\
\begin{align}
	\Delta i &= -G^\prime \Delta Z \cdot U - C^\prime \Delta Z \cdot \dot
	U\nonumber\\
	\boxed{\frac{\delta i}{\delta Z}=-G^\prime U - C^\prime \frac{\delta u}{\delta
	t}}\label{leitungstheorie:wellengleichung2}
\end{align}
Einsetzen, ergibt Telegrafengleichungen\\
% \begin{align}
% 	\frac{\delta^2u}{\delta z^2}=R^\prime G^\prime U+R^\prime C^\prime
% 	\frac{\delta u}{\delta t} +L^\prime G^\prime\frac{\delta u}{\delta t}+L^\prime
% 	C^\prime \frac{\delta^2 u}{\delta t^2}\\
% 	\frac{\delta^2i}{\delta z^2} = R^\primeG^\prime i+\left(R^\prime
% 	C^\prime+L^\prime G^\prime\right)\\
% \end{align}
Ansatz:
\begin{align}
	u(t,z) &= Re \{ \underline{\hat U}(z)\cdot e^{j\omega t} \} \nonumber\\
	u(t,z) &= Re \{ \underline{\hat I}(z)\cdot e^{j\omega t} \} \nonumber
\end{align}
angewandt auf \ref{leitungstheorie:wellengleichung1} und
\ref{leitungstheorie:wellengleichung2}\\
\begin{align}
	\frac{d\underline{U}}{dZ}=-R^\prime\cdot\underline{I}-j\omega
	L^\prime\cdot\underline{I}=- \underbrace{\left(R^\prime+j\omega
	L^\prime\right)}_{\underline{Z}^\prime=\text{Belag der
	Längsimpedanz}}\cdot\underline{I}\label{leitungstheorie:angewandt1}\\
	\frac{d\underline{I}}{dZ}=-G^\prime\cdot\underline{U}-j\omega
	C^\prime\cdot\underline{U}=- \underbrace{\left(G^\prime+j\omega
	C^\prime\right)}_{\underline{Y}^\prime=\text{Belag der
	Queradmittanz}}\cdot\underline{U}\label{leitungstheorie:angewandt2}
\end{align}
\ref{leitungstheorie:angewandt1} ableiten, \ref{leitungstheorie:angewandt2}:
\begin{align}
	\frac{d^2\underline{U}}{dz^2}&=-\left(R^\prime+j\omega
	L^\prime\right)\frac{d\underline{I}}{d\underline{Z}}\nonumber\\
	\frac{d^2\underline{U}}{dz^2}&=\underbrace{\left(R^\prime+j\omega
	L^\prime\right)}_{\underline{Z}^\prime}\underbrace{\left(G^\prime+j\omega
	C^\prime\right)}_{\underline{Y}^\prime}\underline{U}=\underline{\gamma}^2\underline{U}\label{leitungstheorie:telegrafengleichung1}\\
	\frac{d^2\underline{I}}{dz^2}&=\left(\ldots\right)\left(\ldots\right)\underline{I}=\underline{\gamma}^2\underline{I}\label{leitungstheorie:telegrafengleichung2}
\end{align}
\ref{leitungstheorie:telegrafengleichung1} und
\ref{leitungstheorie:telegrafengleichung2} sind die Telegrafengleichungen im
Frequenzbereich.\\
\begin{align}
	\frac{d^2\underline{U}(z)}{dz^2}-\underline{\gamma}^2\underline{U}(z)&=0\nonumber\\
	\underline{U}(z)&=\underline{U}\cdot e^{pt}&&\text{Ansatz, }(p \neq
	j\omega)\nonumber\\
	p^2\cdot
	\underline{U}e^{pt}-\underline{\gamma}\underline{U}e^{pt}&=0\nonumber\\
	\underline{U}e^{pt}\left(p^2-\underline{\gamma}^2\right)&=0\nonumber\\
	p^2-\underline{\gamma}^2&=0)&&\text{Charakteristische Gleichung}\nonumber\\
	p&=\sqrt{\underline{\gamma}^2}\nonumber
\end{align}
\begin{tabular}{lll}
$\gamma$ & Ausbreitungskonstante \\
$\gamma$ & $= \alpha +j\beta$ \\
$\alpha$ & Dampf.-Belag & $[\alpha]=Np/m$\\
$1Np$ & $=$1 Neper & $\mathrel{\widehat{=}}$ Dämpfung um Faktor
$e^1=2.718\mathrel{\widehat{=}}8.686dB$\\
$\beta$ & Phasenbelag & $[\beta]=rad/m$\\
\end{tabular}


