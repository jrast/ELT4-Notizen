\section{Leitungstheorie}
\index{Leitungstheorie}
Ebook Strauss\\
Zur Einführung:\\
\begin{figure}[!h]
\centering
\subfloat[Schema]{
	\tikzexternaldisable
\begin{circuitikz} [scale=2, european]
	\ctikzset{bipoles/length=1.2cm}
	\draw (0,0) to [R=$50\Omega$] (1,0) to [cspst=$0$] (1.9,0) to (5,0) to [R=$25\Omega$] (5,-1)
	to (0,-1)
	(0,0) to [V=$12V$] (0,-1);
	\draw node at (2.5,-0.5) {$\underline{Z}_0 50\Omega$};
	%Spannungspfeile
	\draw [red] (2,0) to[open, v>=$\underline{U}_1$] (2,-1);
	\draw [blue] (3.5,0) to[open, v>=$\underline{U}_2$] (3.5,-1);
	\draw [green] (5,0) to[open, v>=$\underline{U}_3$] (5,-1);
	%Strompfeile
\end{circuitikz}
%\tikzexternalenable
	\label{fig:leitungstheorie:bsp1:schema}}\\
\subfloat[Spannung U]{
	\begin{tikzpicture}[thick]
	\draw [->] (0,0) -- (5,0) node[anchor=west] {t};
	\draw [->] (0,0) -- (0,4) node[anchor=south] {u};
	\draw [green] (-0.2,3) -- (4,3) -- (4,2) -- (5,2);
	\draw [black] (-0.2,0) -- (1,0) -- (1,3) -- (3,3) -- (3,2) -- (5,2);
	\draw [red] (-0.2,0) -- (2,0) -- (2,2) -- (5,2);
	\draw (0.2,1) -- (-0.2,1) node[anchor=east] {2V};
	\draw (0.2,2) -- (-0.2,2) node[anchor=east] {4V};
	\draw (0.2,3) -- (-0.2,3) node[anchor=east] {6V};
	\draw (1,0.2) -- (1,-0.2) node[anchor=north] {T/2};
	\draw (2,0.2) -- (2,-0.2) node[anchor=north] {T};
	\draw (3,0.2) -- (3,-0.2) node[anchor=north] {3T/2};
	\draw (4,0.2) -- (4,-0.2) node[anchor=north] {2T};
\end{tikzpicture}
	\label{fig:leitungstheorie:bsp1:u}
}
\qquad
\subfloat[Strom I]{
	\begin{tikzpicture}[thick]
	\draw [->] (0,0) -- (5,0) node[anchor=west] {t};
	\draw [->] (0,0) -- (0,4) node[anchor=south] {i};
	\draw [green] (-0.2,2) -- (4,2) -- (4,3) -- (5,3);
	\draw [black] (-0.2,0) -- (1,0) -- (1,2) -- (3,2) -- (3,3) -- (5,3);
	\draw [red] (-0.2,0) -- (2,0) -- (2,3) -- (5,3);
	\draw (0.2,1) -- (-0.2,1) node[anchor=east] {80mA};
	\draw (0.2,2) -- (-0.2,2) node[anchor=east] {120mA};
	\draw (0.2,3) -- (-0.2,3) node[anchor=east] {160mA};
	\draw (1,0.2) -- (1,-0.2) node[anchor=north] {T/2};
	\draw (2,0.2) -- (2,-0.2) node[anchor=north] {T};
	\draw (3,0.2) -- (3,-0.2) node[anchor=north] {3T/2};
	\draw (4,0.2) -- (4,-0.2) node[anchor=north] {2T};
\end{tikzpicture}
	\label{fig:leitungstheorie:bsp1:i}
}\\
\caption{U und I im Beispiel, T ist Laufzeit für die Länge}
\label{fig:leitungstheorie:bsp}
\end{figure}
\begin{align}
	\Delta Z &= -R^\prime\cdot\Delta Z \cdot i - L^\prime\Delta Z
	\cdot \dot i\nonumber\\
	\frac{\Delta U}{\Delta Z} &= \left. -R^\prime\cdot i - L^\prime\dot i \right|
	\Delta Z \rightarrow 0\nonumber\\
	\boxed{\frac{\delta u}{\delta z}=-R^\prime i- L^\prime \frac{\delta i}{\delta
	t}}\label{leitungstheorie:wellengleichung1}
\end{align}
Korrektur Buch Seite 61, Formel 3.3, Untere Formel u durch i ersetzen\\
\begin{align}
	\Delta i &= -G^\prime \Delta Z \cdot U - C^\prime \Delta Z \cdot \dot
	U\nonumber\\
	\boxed{\frac{\delta i}{\delta Z}=-G^\prime U - C^\prime \frac{\delta u}{\delta
	t}}\label{leitungstheorie:wellengleichung2}\\
	\gamma = \sqrt{\left(R^\prime+j\omega L\right)\left(G^\prime+j\omega
	C\right)}\nonumber
\end{align}
Einsetzen, ergibt Telegrafengleichungen\\
% \begin{align}
% 	\underline{U}(z)=\underline{U}_{Vo}e^{-\underline{\gamma}
% 	z}+\underline{U}_{ro}e^{\underline{\gamma}Z}\nonumber\\
% 	\underline{I}(z)=\underline{I}_{Vo}e^{-\underline{\gamma}
% 	z}+\underline{I}_{ro}e^{\underline{\gamma}Z}\nonumber\\
% 	\underline{Z}_0=\sqrt{\frac{R^\prime+j\omega L^\prime}{G^\prime+j\omega
% 	C^\prime}}=\sqrt{\frac{Z^\prime}{C^\prime}}\nonumber
% \end{align}
% \begin{align}
% 	\frac{\delta^2u}{\delta z^2}=R^\prime G^\prime U+R^\prime C^\prime
% 	\frac{\delta u}{\delta t} +L^\prime G^\prime\frac{\delta u}{\delta t}+L^\prime
% 	C^\prime \frac{\delta^2 u}{\delta t^2}\\
% 	\frac{\delta^2i}{\delta z^2} = R^\primeG^\prime i+\left(R^\prime
% 	C^\prime+L^\prime G^\prime\right)\\
% \end{align}
Ansatz:
\begin{align}
	u(t,z) &= Re \{ \underline{\hat U}(z)\cdot e^{j\omega t} \} \nonumber\\
	u(t,z) &= Re \{ \underline{\hat I}(z)\cdot e^{j\omega t} \} \nonumber
\end{align}
angewandt auf \ref{leitungstheorie:wellengleichung1} und
\ref{leitungstheorie:wellengleichung2}\\
\begin{align}
	\frac{d\underline{U}}{dZ}=-R^\prime\cdot\underline{I}-j\omega
	L^\prime\cdot\underline{I}=- \underbrace{\left(R^\prime+j\omega
	L^\prime\right)}_{\underline{Z}^\prime=\text{Belag der
	Längsimpedanz}}\cdot\underline{I}\label{leitungstheorie:angewandt1}\\
	\frac{d\underline{I}}{dZ}=-G^\prime\cdot\underline{U}-j\omega
	C^\prime\cdot\underline{U}=- \underbrace{\left(G^\prime+j\omega
	C^\prime\right)}_{\underline{Y}^\prime=\text{Belag der
	Queradmittanz}}\cdot\underline{U}\label{leitungstheorie:angewandt2}
\end{align}
\ref{leitungstheorie:angewandt1} ableiten, \ref{leitungstheorie:angewandt2}:
\begin{align}
	\frac{d^2\underline{U}}{dz^2}&=-\left(R^\prime+j\omega
	L^\prime\right)\frac{d\underline{I}}{d\underline{Z}}\nonumber\\
	\frac{d^2\underline{U}}{dz^2}&=\underbrace{\left(R^\prime+j\omega
	L^\prime\right)}_{\underline{Z}^\prime}\underbrace{\left(G^\prime+j\omega
	C^\prime\right)}_{\underline{Y}^\prime}\underline{U}=\underline{\gamma}^2\underline{U}\label{leitungstheorie:telegrafengleichung1}\\
	\frac{d^2\underline{I}}{dz^2}&=\left(\ldots\right)\left(\ldots\right)\underline{I}=\underline{\gamma}^2\underline{I}\label{leitungstheorie:telegrafengleichung2}
\end{align}
\ref{leitungstheorie:telegrafengleichung1} und
\ref{leitungstheorie:telegrafengleichung2} sind die Telegrafengleichungen im
Frequenzbereich.\\
\begin{align}
	\frac{d^2\underline{U}(z)}{dz^2}-\underline{\gamma}^2\underline{U}(z)&=0\nonumber\\
	\underline{U}(z)&=\underline{U}\cdot e^{pt}&&\text{Ansatz, }(p \neq
	j\omega)\nonumber\\
	p^2\cdot
	\underline{U}e^{pt}-\underline{\gamma}\underline{U}e^{pt}&=0\nonumber\\
	\underline{U}e^{pt}\left(p^2-\underline{\gamma}^2\right)&=0\nonumber\\
	p^2-\underline{\gamma}^2&=0)&&\text{Charakteristische Gleichung}\nonumber\\
	p&=\sqrt{\underline{\gamma}^2}\nonumber
\end{align}
\begin{tabular}{lll}
$\gamma$ & Ausbreitungskonstante \\
$\gamma$ & $= \alpha +j\beta$ \\
$\alpha$ & Dampf.-Belag & $[\alpha]=Np/m$\\
$1Np$ & $=$1 Neper & $\mathrel{\widehat{=}}$ Dämpfung um Faktor
$e^1=2.718\mathrel{\widehat{=}}8.686dB$\\
$\beta$ & Phasenbelag & $[\beta]=rad/m$\\
\end{tabular}

\begin{align}
	\underline{U}(z) &= \underline{U}_{V0}\cdot e^{-\underline{\gamma}
	z}+\underline{U}_{r0}\cdot
	e^{\underline{\gamma}z}=\underline{U}_v(z)+\underline{U}_r(z)\nonumber\\
	%Links vom + grün unterstrichen, rechts blau
	&= \underline{U}_{vo}\cdot e^{-\alpha z}\cdot e^{-j\beta
	z}+\underline{U}_{r0}\cdot e^{\alpha z}\cdot e^{j\beta z}\nonumber
\end{align}
Bild 1 Bild 2\\
\begin{figure}[!h]
\centering
\subfloat[Impedanz in Abhängigkeit von Z]{
	\tikzexternaldisable
\begin{tikzpicture}[thick,scale=2]
	%Achsen
	\draw [->] (-0.2,0) -- (3.2,0) node[anchor=west] {Z};
	\draw [->] (0,-0.2) -- (0,1);
	%Grüne Pfeile
	\draw [green, ->] (0,0) -- (0.2,0.8);
	\draw [green, ->] (1,0) -- (1.4,0.5);
	\draw [green, ->] (2,0) -- (2.2,0.2);
	\draw [green, ->] (3,0) -- (3.2,-0.2);
	%Blaue Pfeile
	\draw [blue, ->] (0,0) -- (-0.2,0.1);
	\draw [blue, ->] (1,0) -- (0.8,-0.1);
	\draw [blue, ->] (2,0) -- (1.8,-0.2);
	\draw [blue, ->] (3,0) -- (2.7,-0.4);
\end{tikzpicture}
\tikzexternalenable

	\label{fig:leitungstheorie:impedanz:abhz}
}
\qquad
\subfloat[Formlen zur Zeigerdarstellung]{
	\tikzexternaldisable
\begin{tikzpicture}[thick,scale=2]
	%Achsen
	\draw [->] (-1,0) -- (1,0) node[anchor=west] {Z};
	\draw [->] (0,-0.2) -- (0,1) node[anchor=south] {Im};
	%Grüne Pfeile
	\draw [green, ->] (0,0) -- (0.2,0.8) node[anchor=south] {$\underline{U}_{U0}$};
	\draw [green, dotted, ->] (0.1,0.7) arc (80:40:0.5);
	%Blaue Pfeile
	\draw [blue, ->] (0,0) -- (-0.4,0.3) node[anchor=south] {$\underline{U}_{r0}$};
	\draw [blue, dotted, ->] (-0.2,0.3) arc (110:170:0.36) node[anchor=north] {$e^{\alpha 2}\cdot e^{\j\beta 2}$};
\end{tikzpicture}
\tikzexternalenable

	\label{fig:leitungstheorie:impedanzen:formeln}
}\\
\caption{Zeigerdarstellung}
\label{fig:leitungstheorie:impedanz}
\end{figure}\\
Eine Umdrehung des Zeigers entspricht einer Wellenlänge $\lambda$\\
\begin{align}
	\Rightarrow \beta z = \beta \lambda = 2 \pi\nonumber\\
	\boxed{\beta=\frac{2\pi}{\lambda}}\nonumber
\end{align}
\textbf{Zurück in den Zeitbereich}
\begin{align}
	u(t,z)&=Re\{\hat{\underline{U}}(z)\cdot e^{j\omega t}\nonumber\\
	%links gelb, rechts vom + blau
	&=Re\{\hat{\underline{U}}_{v0}e^{-\underline{\gamma}z}e^{j\omega
	t}\}+Re\{\hat{\underline{U}}_{r0}e^{\underline{\gamma}z}e^{j\omega
	t}\}\nonumber\\
	U_v(t,z)&=Re\{\underline{\hat{U}}_{v0}\cdot e^{j\varphi_v}\cdot e^{-\alpha
	z}\cdot e^{-j\beta z}\cdot e^{j\omega t}\}\nonumber\\
	%gelb boxed:
	U_v&=\hat{U}_{v0}\cdot e^{-\alpha z}\cdot \cos\left(\omega t-\beta
	z+\varphi_v\right)\nonumber
\end{align}
Bild 3\\
Wie schnell kommt die Welle vorwärts?\\
$\omega t-\beta Z=konst$ (d.h. konstanter Winkel in der cos-Funktion)\\
$\omega\Delta T=\beta\Delta Z$\\
%blau boxed:
$v_p=\frac{\Delta Z}{\Delta t}=\frac{\omega}{\beta}$\\
Skript Seite 64\\
dazu: falls $v=konst. v=\frac{\omega}{\beta}\Rightarrow \omega=v\beta$\\
Bild 4\\
\subsection{Reflexionsfaktor r}
Skript Seite 67\\
Bild 5\\
Am Leitungsende:\\
\begin{align}
	\underline{U}&=\underline{I}\cdot\underline{Z}_E\nonumber\\
	\underline{U}&=\underline{U}_v+\underline{U}_r=\left(\underline{I}_v-\underline{I}_r\right)\cdot\underline{Z}_E\nonumber\\
	\underline{U}_v+\underline{U}_r&=\frac{\underline{Z}_E}{\underline{Z}_0}\left(\underline{U}_v-\underline{U}_r\right)\nonumber\\
	\underline{U}_r\left(1+\frac{\underline{Z}_e}{\underline{Z}_0}\right)&s=\underline{U}_v\left(\frac{\underline{Z}_e}{\underline{Z}_0}-1\right)\nonumber\\
	\underline{r}_E=\frac{\underline{U}_r}{\underline{U}_v}&=\frac{\frac{\underline{Z}_e}{\underline{Z}_0}-1}{\frac{\underline{Z}_e}{\underline{Z}_0}+1}=\frac{\underline{Z}_E-\underline{Z}_0}{\underline{Z}_E+\underline{Z}_0}\nonumber\\
	\text{falls: }\underline{Z}_E=R, \underline{Z}_0&=R_0
	\boxed{r=\frac{R-R_0}{R+R_0}}\nonumber
\end{align}
Skript Seite 68 ff\\
\begin{align}
	\underline{r}_A&=\frac{\underline{U}_{Ar}}{\underline{U}_{Av}}=\frac{\underline{U}_{Er}\cdot
	e^{-\underline{\gamma}  l}}{\underline{U}_{Ev}\cdot e^{\underline{\gamma}
	l}}=\boxed{\underline{r}_E\cdot e^{-2\underline{\gamma}l}}\nonumber
\end{align}
343: $\underline{Z}=\underline{Z}_0\frac{1+r}{1-r}$\\

\subsection{Leitungen als Zweitor}
\begin{figure}[!h]
	\centering
	\begin{tikzpicture}[thick,scale=2]
	\draw (0,0) -- (4,0);
	\draw (4,-1) -- (0,-1);
	\draw [->] (0,0) -- (0.5,0) node[anchor=south] {$\underline{I}_1$};
	\draw [->] (4,0) -- (3.5,0) node[anchor=south] {$\underline{I}_2$};
	\draw [->] (0,-0.1) -- (0,-0.9);
	\node at (0,-0.5) [anchor=east] {$\underline{U}_1$};
	\draw [->] (4,-0.1) -- (4,-0.9);
	\node at (4,-0.5) [anchor=west] {$\underline{U}_2$};
	%Box
	\draw [dashed] (1,-1.5) rectangle (3,0.5);
	\node at (2,-0.5) {Leitung: $l,\underline{\gamma},\underline{Z}_0$}; 
	%Strahl unten
	\draw [->] (0.8,-2) -- (4,-2) node[anchor=west] {Z};
	\draw (1,-1.9) -- (1,-2.1) node[anchor=north] {0};
	\draw (3,-1.9) -- (3,-2.1) node[anchor=north] {$l$};
\end{tikzpicture}
	\caption{Leitung als Zweitor dargestellt}
	\label{fig:leitungstheorie:zweitor}
\end{figure}
Suche $[A]$-Matrix\\
Leitungsgleichungen:\\
\begin{align}
	\underline{U}(z)=\underline{U}_V(z)+\underline{U}_r(z)=\underline{U}_{v0}e^{-\underline{\gamma}z}+\underline{U}_{r0}e^{\underline{\gamma}z}\nonumber\\
	\underline{I}(z)=\underline{I}_V(z)+\underline{I}_r(z)=\frac{\underline{U}_{v0}e^{-\underline{\gamma}z}}{Z_0}-\frac{\underline{U}_{r0}e^{\underline{\gamma}z}}{Z_0}\nonumber\\
	\underline{U}_1=\underline{U}(0), \underline{U}_2=\underline{U}(l),
	\underline{I}_1=\underline{I}(0), \underline{I}_2=-\underline{I}(l)\nonumber\\
	\begin{bmatrix}
 		\underline{U}_1\\
 		\underline{I}_1
	\end{bmatrix}=
	\begin{bmatrix}
 		cosh\underline{\gamma}l & \underline{Z}_0sinh\underline{\gamma}l\\
 		\frac{sinh\underline{\gamma}l}{\underline{Z}_0} & cosh\underline{\gamma}l
	\end{bmatrix}\cdot
	\begin{bmatrix}
		\underline{U}_2\\
		\underline{I}_2
	\end{bmatrix}\nonumber
\end{align}

\subsection{Smith Chart}
Polardarstellung des kompl. Refl-Faktors $\underline{r}$ (mit
Impedanzgitter-Netz)\\
\begin{figure}[!h]
	\centering
 	\tikzexternaldisable
\begin{tikzpicture}[thick]
	\draw [->] (-3.2,0) -- (3.2,0);
	\draw [->] (0,-3.2) -- (0,3.2);
	\draw (0,0) circle (3);
	%Beschriftungen
	\node at (3,0) [anchor=north west] {$1$};
	\node at (0,3) [anchor=south west] {$r=j$};
	\node at (-3,0) [anchor=north east] {$-1$};
	\node at (0,-3) [anchor=north west] {$r=-j$};
	\node at (0,0) [anchor=north west] {$r=0$};
	\node at (-2.5,1.5) [anchor=south east] {$|\underline{r}|=1$};
	%Pfeil
	\draw [->,green] (0,0) -- (1,0.5) node [green,anchor=west] at (1,0.5) {$\underline{r}=r\cdot e^{j\varphi}$};
	\draw [green] (0.8,0) arc (0:20:1) node [anchor=north west] {$\varphi$};
\end{tikzpicture}
\tikzexternalenable
	\caption{Polardarstellung des Smith Chart}
	\label{fig:leitungstheorie:smith}
\end{figure}\\
$z$-Ebene wird abgebildet in $r$-Ebene\\
$\underline{r}=\frac{\underline{Z}-R_0}{\underline{Z}+R_0}$\\
allg: $\underline{w}=\frac{a\underline{Z}+b}{c\underline{Z}+d}$ gebrochen
lineare Fkz (Abb)\\
$\rightarrow$ konform (winkeltreu)\\
$\rightarrow$ kreistreu ("`Kreise"'$\rightarrow$"'Kreise"')\\
\begin{figure}[!h]
\centering
\subfloat[Z-Ebene]{
	\begin{tikzpicture}[thick,dot/.style={fill=red,circle,minimum
size=1pt}]
	%Axen
	\draw [->] (-1,0) -- (4,0) node[anchor=west] {$R$};
	\draw [->] (0,-1) -- (0,4) node[anchor=south] {$jX$};
	\node [dot] at (0,0) {};
	\node [red, anchor=north east] at (0,0) {$Z=0$};
	\node [red, anchor=north east] at (0,-0.5) {$KS$};
	\node [green, anchor=north west] at (1,0) {$R_0$};
	\draw [green, dashed] (0.33,-1) -- (0.33,4);
	\node [green, anchor=south west] at (0.33,0) {$\frac{1}{3}R_0$};
	\draw [green] (1,-1) -- (1,4);
	\draw [blue] (3,-1) -- (3,4);
	\node [blue, anchor=north west] at (3,0) {$3R$};
	%Horizontale:
	\draw (4,1) -- (-0.2,1) node[anchor=east] {$jR_0$};
\end{tikzpicture}
	\label{fig:leitungstheorie:smith:zrz}
}
\qquad
\subfloat[R-Ebene]{
	\begin{tikzpicture}[thick,dot/.style={fill=red,circle,minimum
size=1pt}]
	%einzigste Gerade
	\draw (-2,0) -- (2,0);
	%Kreise
	\draw (0,0) circle (2);
	\draw [green] (0.5,0) circle (1.5);
	\draw [green] (1,0) circle (1);
	\draw [blue] (1.5,0) circle (0.5);
	\draw (0,2) arc (180:270:2);
	\draw (0,-2) arc (180:90:2);
	%Rote bemerkungen
	\draw [red] (-2,0.2) -- (-2,-0.2);
	\draw [red] (-1,0.2) -- (-1,-0.2);
	\draw [red] (0,0.2) -- (0,-0.2);
	\draw [red] (1,0.2) -- (1,-0.2);
	\draw [red] (2,0.2) -- (2,-0.2);
	%Rote beschriftungen
	\node [red,anchor=east] at (-2,0) {$Z=0$};
	\node [red,anchor=east] at (-2,-0.5) {$Kurzschluss$};
	\node [red,anchor=north east] at (-1,0) {$\frac{1}{3}R_0$};
	\node [red,anchor=north east] at (0,0) {$R_0$};
	\node [red,anchor=north east] at (1,0) {$3R_0$};
	\node [red,anchor=west] at (2,0) {$Z\rightarrow\infty$};
	\node [red,anchor=west] at (2,-0.5) {$Leerlauf$};
	\node [anchor=south] at (0,2) {$jR_0$};
	\node [anchor=north] at (0,-2) {$-j$};
\end{tikzpicture}
	\label{fig:leitungstheorie:smith:zrr}
}\\
\caption{Transformation zwischen Z- und R-Ebene}
\label{fig:leitungstheorie:smith}
\end{figure}\\
Wertetabelle\\
\begin{tabular}{|l|l|l|l|l|l|l|}
	\hline
	$\underline{Z}$ & $R_0$ & 0 & $\infty$ & $\frac{1}{3}R_0$ & $3R_0$ & $jR_0$\\
	\hline
	$\underline{r}$ & 0 & -1 & 1 & $-\frac{1}{2}$ & $\frac{1}{2}$ & $j$\\
	\hline
\end{tabular}\\
Mitschrift 31.5.\\
Kompl. r-Ebene:\\
Bild 1\\
